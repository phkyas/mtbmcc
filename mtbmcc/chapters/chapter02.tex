% ****************************************************************************************************
\chapter{Theoretical Foundations}\label{ch:tf}
% ****************************************************************************************************

\begin{flushright}{\slshape    
	When the wind of change blows,\\
	some people build walls,\\
	others build windmills.} \\ \medskip
	--- Chinese Proverb
\end{flushright}

Within this chapter, the theoretical foundations of the thesis at hand are elucidated. At the beginning, the \ac{PaaS} domain, relations to other cloud computing components, as well as cloud computing itself are introduced. Afterwards, the term business model and various points of view in regards to business models are explained. Furthermore, one business model conceptualization which is used throughout this thesis is described at the end of this chapter in detail. 

\section{Platform as a Service}\label{ch:tf:paas}
\begin{flushright}{\slshape 
	In pioneer days they used oxen for heavy pulling,\\
	and when one ox couldn't budge a log,\\
	they didn't try to grow a larger ox.\\
	We shouldn't be trying for bigger computers,\\
	but for more systems of computers.} \\ \medskip
	--- Grace Hopper
\end{flushright}

%\hangbibentry{Armbrust2010}
%\hangbibentry{Mell2011a}
%\hangbibentry{Karabek2011}
%\hangbibentry{Vaquero2009}
%\hangbibentry{Weinhardt2009}
%\hangbibentry{Iyer2010}
%PaaS Papers are missing (Def and so on)

\section{Business Models}\label{ch:tf:bm}
%\hangbibentry{Morris2005}
%\hangbibentry{Osterwalder2005}
%\hangbibentry{Zott2011}
%\hangbibentry{Pateli2004}
%\hangbibentry{Johnson2008}
%Some Paper(s) form Chesbrough (~2)
%\citep{Johnson2008}

\begin{flushright}{\slshape    
	There is nothing quite so useless,\\
	as doing with great efficiency, \\
	something that should not be done at all.} \\ \medskip
	--- Peter F. Drucker
\end{flushright}

\subsection{Origins and Definitions}

Already 1954 Peter F. Drucker proposed the five following questions in order to describe basic business strategies \citep[pp. 49-61]{Drucker1954}:

\begin{enumerate}
	\item What is our business?
	\item Who is the customer?
	\item What is value to the customer?
	\item What will our business be?
	\item What should it be?
\end{enumerate}

However, the term business models itself evolved noticeable within academic journals as late as during the nineties \mycite{Osterwalder2005}{pp. 3-4}{Zott2011}{p. 1022}. As of this writing this paper at hand, there is no generally accepted definition or uniform picture about this concept. According to \citet[p. 726]{Morris2005} and \citet[p. 1022]{Zott2011}, the term business models (at a meta level) has been referred to as an architecture, an assumption, a conceptual tool or model, a description, a design, a framework, a method, a pattern, a plan, a representation, a set, a statement, as well as a structural template. Therefore, a couple of widely used definitions with different points of view are presented below:
	
\begin{quote}{\slshape 
"[A business model is] an architecture for the product, service and information flows, including a description of the various business actors and their roles; and a description of the potential benefits for the various business actors; and a description of the sources of revenues."}
\vspace*{-5pt}
\begin{flushright}
	--- \citet[p. 2]{Timmers1998}
\end{flushright}
\end{quote}

\begin{quote}{\slshape 
"Business models are defined as summary of the value creation logic of an organization or a business network including assumptions about its partners, competitors and customers. They define the business and IS architecture, rules, potential benefits and the sources of revenue."}
\vspace*{-5pt}
\begin{flushright}
	--- \citet[p. 798]{Klueber2000}
\end{flushright}
\end{quote}

\begin{quote}{\slshape 
"A business model depicts the content, structure, and governance of transactions designed so as to create value through the exploitation of business opportunities."}
\vspace*{-5pt}
\begin{flushright}
	--- \citet[p. 511]{Amit2001}
\end{flushright}
\end{quote}

\begin{quote}{\slshape 
"[A business model is the] heuristic logic that connects technical potential with the realization of economic value."}
\vspace*{-5pt}
\begin{flushright}
	--- \citet[p. 529]{Chesbrough2002}
\end{flushright}
\end{quote}

\begin{quote}{\slshape 
"[Business models are] stories that explain how enterprises work. A good business model answers Peter Drucker's age-old questions: Who is the customer? And what does the customer value? It also answers the fundamental questions every manager must ask: How do we make money in this business? What is the underlying economic logic that explains how we can deliver value to customers at an appropriate cost?"}
\vspace*{-5pt}
\begin{flushright}
	--- \citet[p. 2]{Magretta2002}
\end{flushright}
\end{quote}

\begin{quote}{\slshape 
"A business model is a concise representation of how an interrelated set of decision variables in the areas of venture strategy, architecture, and economics are addressed to create sustainable competitive advantage in defined markets."}
\vspace*{-5pt}
\begin{flushright}
	--- \citet[p. 727]{Morris2005}
\end{flushright}
\end{quote}

\begin{quote}{\slshape 
"A business model is a conceptual tool that contains a set of elements and their relationships and allows expressing the business logic of a specific firm. It is a description of the value a company offers to one or several segments of customers and of the architecture of the firm and its network of partners for creating, marketing, and delivering this value and relationship capital, to generate profitable and sustainable revenue streams."}
\vspace*{-5pt}
\begin{flushright}
	--- \citet[p. 10]{Osterwalder2005}
\end{flushright}
\end{quote}
	
\begin{quote}{\slshape 
"[A business model is] a representation of a firm's underlying core logic and strategic choices for creating and capturing value within a value network."}
\vspace*{-5pt}
\begin{flushright}
	--- \citet[p. 202]{Shafer2005}
\end{flushright}
\end{quote}

\begin{quote}{\slshape 
"The particular business concept (or way of doing business) as reflected by the business's core value proposition(s) for customers; its configurated value network to provide that value, consisting of own strategic capabilities as well as other (e.g. outsourced/allianced) value networks; and its continued sustainability to reinvent itself and satisfy the multiple objectives of its various stakeholders."}
\vspace*{-5pt}
\begin{flushright}
	--- \citet[p. 40]{Voelpel2005}
\end{flushright}
\end{quote}

\begin{quote}{\slshape 
"A business model \ldots consists of four interlocking elements [\ac{CVP}, profit formula, key resources, and key processes] that, taken together, create and deliver value."}
\vspace*{-5pt}
\begin{flushright}
	--- \citet[p. 52]{Johnson2008}
\end{flushright}
\end{quote}

\subsection{Conceptualization}\label{ch:sota:bmc}

Based on an analysis of definitions for business models in literature (cf. Section \ref{ch:tf:bm}), the definition of \citet{Johnson2008} was taken as a basis for the research presented in this paper. According to \citet[p. 52]{Johnson2008}, "A business model \ldots consists of four interlocking elements that, taken together, create and deliver value". These four elements are denoted as follows: \ac{CVP}, Profit Formula, Key Resources and Key Processes. 

The \ac{CVP} is considered as the most crucial part based on its main objective to satisfy customer needs. In order to understand the customer needs, the following key questions should be answered: Who are my target customers?, What is their job to be done (problem)?, and How can their problems be solved (offering)? The interrelation between the job to be done and the corresponding offer as well as the value of the \ac{CVP} are described by  \citet[p. 52]{Johnson2008} in the following way, "The more important the job is to the customer, the lower the level of customer satisfaction with current options for getting the job done, and the better your solution is than existing alternatives at getting the job done (and, of course, the lower the price), the greater the CVP."'

How a company is capturing value for itself based on the offering to its customer is another key element within each business model -- named here as profit formula, consisting of the sub elements revenue model, cost structure, margin model, and resource velocity.

Assets which are essential to create and deliver the above described customer value proposition are summarized as key resources. According to \citet[p. 53]{Johnson2008}, these key resources include people, technology, products, facilities, equipment, channels, as well as the company's brand respectively reputation.

Processes in combination with the assets (key resources) are necessary to deliver the value proposition to the company's customers. These processes need to be efficient, repeatable, and scalable. Key processes include repeating operations -- training, development, manufacturing, budgeting, planning, sales, and services -- as well as the company's rules, metrics, and norms \citep[p. 53]{Johnson2008}.

In summary, within the business model conceptualization proposed by \citet[p. 54]{Johnson2008} the customer value proposition and the profit formula encompass the value description for both parties, the target customers as well as the company itself. The value creation and delivery is explained by the two business model elements key resources and key processes. In Table \ref{bm:concept} the business model conceptualization is graphical depicted.

% ****************************************************************************************************
% Business Model Concept
% ****************************************************************************************************
\begin{longtable}{|L{\column}|L{\column}|L{\column}|L{\column}|}
	
	\hline
	\endfirsthead
	\hline  
	\multicolumn{4}{|l|}{\textit{continued from previous page (\nameref{bm:concept})}}\\ 
	\hline
	\endhead
	\hline
	\multicolumn{4}{|r|}{\textit{continued on next page}}\\
	\hline
	\endfoot
	\hline
	\caption[Business Model Conceptualization]{Business Model Conceptualization adapted from \citet[p. 54]{Johnson2008}}
	\label{bm:concept}
	\endlastfoot
	
	\multicolumn{4}{|>{\columncolor[gray]{0.95}}c|}{\large \textbf{Customer Value Proposition (CVP)}}\\ \hline
	
	\textbf{Target customer} &
	\textbf{Job to be done} to solve an important problem or fulfill an important need for the target customer &
	\multicolumn{2}{|L{\columnT}|}{\textbf{Offering}, which satisfies the problem or fulfills the need. This is defined not only by what is sold but also by how it's sold.} \\ \hline

% KEY RESOURCES
\multicolumn{2}{|L{\columnT}|}{\large \textbf{Key Resources} \normalsize needed to deliver the customer value proposition profitably. Might include:
\begin{itemize}[leftmargin=*, parsep=0pt, topsep=0pt, itemsep=0pt]
	\item \textbf{People}
	\item \textbf{Technology, products}
	\item \textbf{Equipment}
	\item \textbf{Information}
	\item \textbf{Channels}
	\item \textbf{Partnerships, alliances}
	\item \textbf{Brand}\vspace{-\baselineskip} 
\end{itemize}
} &
\multicolumn{2}{|L{\columnT}|}{\large \textbf{Key Processes} \normalsize as well as rules, metrics, and norms, that make the profitable delivery of the customer value proposition repeatable and scalable. Might include:
\begin{itemize}[leftmargin=*, parsep=0pt, topsep=0pt, itemsep=0pt]
	\item \textbf{Processes:} design, product development, sourcing, manufacturing, marketing, hiring and training, IT
	\item \textbf{Rules and metrics:} margin requirements for investment, credit terms, lead times, supplier terms
	\item \textbf{Norms:} opportunity size needed for investment, approach to customers and channels\vspace{-\baselineskip} 
\end{itemize}} \\ \hline

% PROFIT FORMULA
\multicolumn{4}{|L{\columnF}|}{\large \textbf{Profit Formula} \normalsize
\begin{itemize}[leftmargin=*, parsep=0pt, topsep=0pt, itemsep=0pt]
		\item \textbf{Revenue model:} How much money can be made: price x volume. Volume can be thought of in terms of market size, purchase frequency, ancillary sales, etc.
		\item \textbf{Cost structure:} How costs are allocated: includes cost of key assets, direct costs, indirect costs, economies of scale.
		\item \textbf{Margin model:} How much each transaction should net to achieve desired profit levels. 
		\item \textbf{Resource velocity:} How quickly resources need to be used to support target volume. Includes lead times, throughput, inventory turns, asset utilization, and so on.\vspace{-\baselineskip} 
\end{itemize} 
}\\ 
\end{longtable}





The main focus at all conducted case studies was rather how the business model appears on the market and is experienced by customers than how the business model is implemented by a specific company or to describe the internal view of the business model. For this reason, the market view elements -- the \ac{CVP} and, in parts, the profit formula -- are mainly considered, even though the non-market view elements -- key resources and key processes -- are mentioned as far as possible. Moreover, the focus on market view elements allowed to solely utilize secondary data for the conducted analysis of \ac{PaaS} providers' business models -- for instance reliable, public data or documentations, both providing comprehensive information about the business model under investigation. Information concerning the internal business model elements -- key resources, key processes, and, in parts, the profit formula -- is difficult to obtain, due to the fact, that this information is in most cases kept confidential, for instance, the cost structure as well as process or workflow descriptions. Thus, the following presented business models mainly focus on the market view elements.