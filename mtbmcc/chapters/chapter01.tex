% ****************************************************************************************************
\chapter{Introduction}\label{ch:intro}
% ****************************************************************************************************

\section{Problem and Motivation}\label{ch:intro:mo}

% evtl. ein Paragraph mit Related Research, sodass die Intro etwas länger ist...

Cloud computing and its inherent components \acf{IaaS}, \acf{PaaS}, as well as \acf{SaaS} introduce a new software paradigm within the \acf{IT} domain. The way cloud-based solutions are developed, deployed, managed, distributed, used, and priced is fundamentally redefined. Eminently, the intermediary \ac{PaaS} layer provides new possibilities, particularly with regard to the way software is developed and deployed. While the concepts of \ac{IaaS} and \ac{SaaS} are quite mature and widely used, the \ac{PaaS} layer is still immature and considered to be at the so-called \textit{"Peak of Inflated Expectations"} \citep[p. 5]{Smith2012}. However, the public \ac{PaaS} market revenue is expected to grow from \$2.6 billion in 2011 to \$9.8 billion in 2016 \citep[p. 22]{Hendrick2012a}.

Nevertheless, almost all \ac{PaaS} offers are characterized by two-sided or even multi-sided business models. Broadly speaking, \ac{PaaS} providers address the two customer segment groups of service providers and service consumers, which are in general disjoint. Flourishing platforms need to address both in order to be successful. This aspect is also often referred to as 'platform ecosystem' or 'platform strategies'. \textit{"Who wins and who loses \ldots [the competition among \ac{PaaS} providers] is not simply a matter of who has the best technology or the first product. It is often who has the best platform strategy and the best ecosystem to back it up"} \citep[p. 34]{Cusumano2010}.

In addition to involving diverse customer segments, \textit{"modern business models are increasingly complex, particularly those with strong ICT [\acl{ICT}] and ebusiness components. The relationship between the different elements of a business model and the decisive success factors are not always immediately observable"} \citep[p. 14]{Osterwalder2005}. Even a small change in a single business model element will affect the other elements as well as the whole system. As a first step, these business model elements need to be revealed and characterized at a meta level. Then the relationships between these elements can be investigated in order to understand the complex dynamics behind \ac{PaaS} business models.

In the recent past, various researchers have worked on different conceptualizations and approaches to modeling platform-based markets and business models. A number of recent publications deal with identifying promising business and technology platform strategies for two-sided markets, elements of platform leadership, and the best practices how to become a platform leader. These results were mainly designed for general applicability (i.e. not market- or industry-specific), although a few were tailored to the \ac{IT} and even \ac{PaaS} industry. Moreover, several researchers have investigated the interrelationships between business model elements and demonstrated their results by means of qualitative representations. This thesis extends existing approaches by modeling the generic \ac{PaaS} business model qualitatively  as well as quantitatively in contrast to prior research.

\section{Research Questions and Objectives}\label{ch:intro:rq}

This thesis investigates how \ac{PaaS} platforms are adopted and what are high-leverage policies in the platform adoption process. This aim engenders the following main research question:

\begin{MRQ}\label{mrq}
How to design Platform as a Service business models in order to achieve a high adoption rate?
\end{MRQ}

In order to answer the main research question, three subordinate research questions were derived which are more precise and focus on a narrower domain. In what follows, these sub-questions are formulated and discussed.

The current \ac{PaaS} market is distinguished by its immature nature and a remarkable provider fluctuation. Moreover, current \ac{PaaS} offers range from simple web application platforms to extensive platforms supporting various use cases. As a first step it is therefore inevitable to assess the current state of \ac{PaaS} business models. This leads to the first sub-research question:

\begin{SRQ}\label{srq1}
How can Platform as a Service business models be classified?
\end{SRQ}

Business models, in particular those with several customer segments and crucial network effects, are a complex field of study. Their inherent interrelationships and interdependencies are not always immediately observable. The impact of these relationships on the rate of platform adoption is of especially great importance. Hence the second sub-research question reads as follows:

\begin{SRQ}\label{srq2}
How are the different elements of a Platform as a Service business model interrelated, especially with regard to the platform adoption?
\end{SRQ}

Furthermore, with a view to identifying high-leverage interventions and policies, it is highly valuable to investigate the impact of even small changes in single business model elements on the overall platform adoption process. As a precondition, the relationships between those elements need to be described mathematically in a rigorous manner. Consequently the third sub-research question arises:

\begin{SRQ}\label{srq3}
How can the relationships between Platform as a Service business model elements which have been revealed be quantified, and how can the platform adoption be simulated?
\end{SRQ}

The desired outcome of the present thesis is threefold. First, it is aimed to develop a classification scheme, answering the first sub-research question. The other two sub-research questions are addressed utilizing the concepts of system dynamics. Second, then a qualitative model in form of a \acf{CLD} is developed to illustrate the interdependencies within \ac{PaaS} business models, which constitutes the answer to the second sub-research question. Third, this model is developed further into a quantitative model in form of a \acf{SFD}. This represents the result of the third sub-research question.
	
\section{Research Methodology and Structure}\label{ch:intro:met}

The thesis present follows a \acf{DSR} methodology (\citealp{March1995}; \citealp{Hevner2004}; \citealp{Hevner2007}; and \citealp{Peffers2007}). According to \citet[p. 77]{Hevner2004}, \textit{"[\ac{DSR}] creates and evaluates IT artifacts intended to solve identified organizational problems"}. Unsolved problems are addressed in unique or innovative ways and artifact improvements are developed for problems that have already been solved \citep[p. 81]{Hevner2004}. The overall goal of \ac{DSR} is its utility \citep[p. 80]{Hevner2004}. \citet[pp. 255-262]{March1995} defined a two-dimensional framework for \ac{IT} research. The first dimension describes four different research activities: (1) build, (2) evaluate, (3) theorize, and (4) justify. Possible outcomes are characterized in the second dimension: (1) representational constructs, (2) models, (3) methods, and (4) instantiations. This framework was further developed by \citet[pp. 78-81]{Hevner2004} and seven corresponding \ac{DSR} guidelines were introduced \citep[pp. 82-90]{Hevner2004}. Moreover, \citet[pp. 87-92]{Hevner2007} emphasized a three-cycle view of \ac{DSR} -- (1) relevance cycle, (2) design cycle, and (3) rigor cycle. First, the relevance cycle is concerned with providing a solution for a relevant and currently existing problem. Second, the design cycle represents the core of \ac{DSR} and interrelates the artifact construction and evaluation activities. Finally, the rigor cycle serves to ground \ac{DSR} for both design cycle activities on scientific knowledge, also known as the knowledge base, as well as to add new knowledge to the knowledge base. All three cycles are by their nature iterative processes. The design cycle especially needs to be run through multiple times, the feedback of each iteration being taken into account at the next run. However, this process should terminate at a certain point. Also, the results developed in this thesis, viz. the classification scheme and the qualitative as well as quantitative models, were arrived at through an iterative approach, even though only the result of the final iteration is included. The interim results were discussed with the supervisors of this thesis, several experts, and a focus group.

%Without additional value
\begin{comment}
	\begin{table}[t]
		\caption[Design Science Research Guidelines]{Design Science Research Guidelines adapted from \citet[p. 83]{Hevner2007}}
		\label{tab:dsrg}
		\centering
		\begin{tabular}{L{.25\textwidth}L{.65\textwidth}}
				\toprule 
				\small \textbf{Guideline} &\small  \textbf{Description} \\ \midrule
				\small G1: Design as an Artifact & 
				\small Design-science research must produce a viable artifact in the form of a construct, a model, a method, or an instantiation.\\ \midrule
				\small G2: Problem Relevance&
				\small The objective of design-science research is to develop technology-based solutions to important and relevant business problems.\\ \midrule
				\small G3: Design Evaluation&
				\small The utility, quality, and efficacy of a design artifact must be rigorously demonstrated via well-executed evaluation methods.\\ \midrule
				\small G4: Research Contributions&
				\small Effective design-science research must provide clear and verifiable contributions in the areas of the design artifact, design foundations, and/or design methodologies.\\ \midrule
				\small G5: Research Rigor&
				\small Design-science research relies upon the application of rigorous methods in both the construction and evaluation of the design artifact.\\ \midrule
				\small G6: Design as a Search Process&
				\small The search for an effective artifact requires utilizing available means to reach desired ends while satisfying laws in the problem environment.\\ \midrule
				\small G7: Communication of Research&
				\small Design-science research must be presented effectively both to technology-oriented as well as management-oriented audiences.\\ \bottomrule
		\end{tabular}
	\end{table}
\end{comment}

\citet[pp. 52-56]{Peffers2007} have developed a six-stage \ac{DSR} process model for the purpose of conducting \ac{DSR} in a rigorous manner. The six activities are (1) identify problem \& motivate, (2) define objectives of a solution, (3) design \& development, (4) demonstration, (5) evaluation, and (6) communication. These were applied here as follows (cf. Figure \ref{fig:dsrm}):

\begin{enumerate}
	\item Identify problem \& motivate: As mentioned above, cloud computing represents an emerging field, and especially the intermediary \ac{PaaS} layer is located at the so-called \textit{"Peak of Inflated Expectations"} \citep[p. 5]{Smith2012}. Moreover, nowadays a superior product (i.e. the platform) in itself is no longer sufficient to operate a successful business. Hence, \ac{PaaS} providers also need to consider the platform strategy as well as the platform ecosystem. Consequently, complex interdependencies are inherent in platform-based business models and these relationships are not always immediately observable. A model or artifact which clearly reveals the above-mentioned problem currently does not exist, so that this research is considered a problem-centered initiation. This first \ac{DSR} activity is discussed in Section \ref{ch:intro:mo}.
	\item Define objectives of a solution: The desired objective of this research is threefold. First, a classification scheme is developed in order to classify \ac{PaaS} business models and their main components. Based on this scheme, the crucial interdependencies within \ac{PaaS} business models with regard to platform adoption are investigated. This process results in the qualitative and quantitative models, which represent the second and third objectives. The corresponding research questions and the above-mentioned objectives are discussed in Section \ref{ch:intro:rq}.
	\item Design \& development: Based on 23 explorative case studies in line with \citet{Eisenhardt1989} and \citet{Yin2008}, the classification scheme for \ac{PaaS} business models is elaborated. For this process, the classification methodology introduced by \citet{Fettke2003} is used. Subsequently, the qualitative as well as quantitative models are developed. The former aims to exhibit the interdependencies within the platform-based business model, whereas the latter makes it possible to simulate the platform adoption. Both models were developed using the concept of systems dynamics, and following the approach described by \citet{Sterman2000,Sterman2001}. As previously mentioned, the classification scheme and both models were arrived at through an iterative approach. This third \ac{DSR} activity is discussed within the Chapters \ref{ch:sota}-\ref{ch:sfd}.
	\item Demonstration: All 23 investigated \ac{PaaS} providers are classified by means of the previously elaborated classification scheme, whereby the usability of this classification scheme is demonstrated. In addition, both the qualitative (\ac{CLD}) and quantitative (\ac{SFD}) system dynamics models are illustrated, using descriptions and graphs. This fourth \ac{DSR} activity is discussed in Section \ref{ch:sota:cPaaS} as well as in Chapters \ref{ch:cld} and \ref{ch:sfd}.
	\item Evaluation: As already mentioned, all three desired outcomes are developed and evaluated with an iterative approach throughout the research process, using a focus group and expert interviews. The results ultimately introduced always represent the final iteration. As a \ac{PoC}, the developed quantitative model is applied in a simulation of \ac{PaaS} business models. This fifth \ac{DSR} activity is discussed in Section \ref{ch:sota:cm} as well as in Chapters \ref{ch:cld}-\ref{ch:poc}.
	\item Communication: The document at hand represents the first publication of this research. Moreover, it is planned to submit the results of this thesis as a conference paper either for the 47th \ac{HICSS} or for the 5th \ac{ICOSB}.
\end{enumerate}

The big picture of this thesis is provided in the following Figure \ref{fig:dsrm}, which illustrates the \ac{DSR} methodology process according to \citet{Peffers2007} as well as the structure of the thesis.

\newpage

\ifthenelse{
	\isodd{\thepage}
}{%odd
	\begin{figure}[t]
		\centering
		\input{gfx/designScienceResearchMethodologyOdd}
		\caption[Design Science Research Methodology -- Process Model]{Design Science Research Methodology -- Process Model, adapted from \citet[p. 54]{Peffers2007}}
		\label{fig:dsrm}
	\end{figure}
}{%even
	\begin{figure}[t]
		\centering
		\input{gfx/designScienceResearchMethodologyEven}
		\caption[Design Science Research Methodology -- Process Model]{Design Science Research Methodology -- Process Model, adapted from \citet[p. 54]{Peffers2007}}
		\label{fig:dsrm}
	\end{figure}
}

