% ****************************************************************************************************
\chapter{Introduction}\label{ch:intro}
% ****************************************************************************************************

\section{Problem and Motivation}\label{ch:intro:mo}

% evtl. ein Paragraph mit Related Research, sodass die Intro etwas länger ist...

Cloud computing and its inherent components \acf{IaaS}, \acf{PaaS}, as well as \acf{SaaS} introduce a new software paradigm within the \acf{IT} domain. The way how cloud-based solutions are developed, deployed, managed, distributed, consumed, and priced is fundamentally redefined. Eminently the intermediary \ac{PaaS} layer provides new possibilities particularly with regards to how software is developed and deployed. Whereas the concepts \ac{IaaS} and \ac{SaaS} are rather mature and widely used, the \ac{PaaS} layer is still immature and considered at the so-called \textit{"Peak of Inflated Expectations"} \citep[p. 5]{Smith2012}. However, it is expected that the public \ac{PaaS} market revenue is growing from \$2.6 billion in 2011 to \$9.8 billion in 2016 \citep[p. 22]{Hendrick2012a}.

Nevertheless, almost all \ac{PaaS} offers are characterized by two- or even multi-sided business models. Simplified, \ac{PaaS} providers' address the two general disjoint customer segment groups of service providers and service consumers. Flourishing platforms need to address both customer segment groups in order to be successful. This aspect is also often referred to as platform ecosystem respectively platform strategies. \textit{"Who wins and who loses \ldots [the competition among \ac{PaaS} providers'] is not simply a matter of who has the best technology or the first product. It is often who has the best platform strategy and the best ecosystem to back it up"} \citep[p. 34]{Cusumano2010}.

In addition to the diverse customer segments, \textit{"modern business models are increasingly complex, particularly those with strong ICT [\acl{ICT}] and ebusiness components. The relationship between the different elements of a business model and the decisive success factors are not always immediately observable"} \citep[p. 14]{Osterwalder2005}. Even small changes of any elements of a business model affect the other elements as well as the whole systems. As a first step these business model elements at a meta level need to be revealed and characterized. Thereafter the relationships between the identified elements can be investigated so as to understand the complex dynamics behind \ac{PaaS} business models.

In the recent past, various academics have researched in the field of platform-based markets and about business models (conceptualization and modeling). Promising business and technology platform strategies for two-sided markets, elements of platform leadership, and best practices to become a platform leader have been identified and published. These results were mainly aimed for a general-purpose (i.e. not market- or industry-specific), but occasional dedicated to the \ac{IT} and even \ac{PaaS} industry. Moreover, several researchers investigated the interrelationships between business model elements and demonstrated the results by means of qualitative representations. To go one step further, within this thesis the generic \ac{PaaS} business model is modeled qualitative and -- in contrast to prior research -- also quantitative.

\section{Research Questions and Objectives}\label{ch:intro:rq}

Within this thesis it is researched how \ac{PaaS} platforms are adopted and what are high leverage policies in the platform adoption process. This aim leads to the following main research question:

\begin{MRQ}\label{mrq}
How to design Platform as a Service business models in order to achieve a high adoption rate?
\end{MRQ}

In order to answer the main research question, three sub-research questions have been derived which are more precisely and focus on a narrower domain. These sub-research questions are discussed and named in the following.

The current \ac{PaaS} market is distinguished by its immature nature and a remarkable provider fluctuation. Moreover, current \ac{PaaS} offers range from simple web application platforms to extensive platform supporting various use cases. As a first step it is therefore inevitable to assess the current state of \ac{PaaS} providers' business models and thus leads to the first sub-research question:

\begin{SRQ}\label{srq1}
How can business models of Platform as a Service providers' be classified?
\end{SRQ}

Business models -- in particular those with several customer segments and crucial network effects -- are a complex field of study. The inherent interrelationships and interdependencies are not always trivially observable. Especially the impact of these relationships in regards to the platform adoption is of great value. Hence the second sub-research question reads as follows:

\begin{SRQ}\label{srq2}
How are the identified Platform as a Service business model elements -- especially in regards to the platform adoption -- interrelated?
\end{SRQ}

Furthermore, the investigation of even small changes in single business model elements on the overall platform adoption process is highly valuable in order to identify high-leverage interventions and policies. As a precondition, the relationships between those elements need to be described mathematically in a stringent manner. Consequently the third sub-research question arises:

\begin{SRQ}\label{srq3}
How can the revealed relationships between Platform as a Service business model elements be quantified and the platform adoption simulated?
\end{SRQ}

The desired outcome of the thesis at hand is threefold. First, it is aimed to develop a classification scheme as result for the sub-research question 1. By utilizing the concept of system dynamics the other two sub-research questions are attacked. Second, a qualitative model (\ac{CLD}) should illustrate the interdependencies within\ac{PaaS} business models and constitutes the answer for the sub-research question 2. This model is further developed into a quantitative model (\ac{SFD}) and represents the result of the sub-research question 3 as well as the final outcome of this thesis.
	
\section{Research Methodology and Structure}\label{ch:intro:met}

The thesis at hand follows a \ac{DSR} methodology (\citealp{March1995}; \citealp{Hevner2004}; \citealp{Hevner2007}; and \citealp{Peffers2007}). According to \citet[p. 77]{Hevner2004}, \textit{"[\ac{DSR}] creates and evaluates IT artifacts intended to solve identified organizational problems."} Whereas unsolved problems are addressed in unique or innovative ways and artifact improvements are developed for already solved problems \citep[p. 81]{Hevner2004}. The overall goal of \ac{DSR} is its utility \citep[p. 80]{Hevner2004}.
\citet[pp. 255-262]{March1995} defined a two dimensional framework for research in \ac{IT}. The first dimension describes four different research activities: (1) build, (2) evaluate, (3) theorize, and (4) justify. Possible outcomes are characterized in the second dimension: (1) representational constructs, (2) models, (3) methods, and (4) instantiations. This framework was further developed by \citet[pp. 78-81]{Hevner2004} and seven corresponding \ac{DSR} guidelines were introduced \citep[pp. 82-90]{Hevner2004}. Moreover, \citet[pp. 87-92]{Hevner2007} emphasized a three cycle view of \ac{DSR} -- (1) relevance cycle, (2) design cycle, and (3) rigor cycle. First, the relevance cycle stresses to provide a problem solution for a relevant and existing problem. Second, the design cycle represents the core of \ac{DSR} and interrelates the artifact construction as well as evaluation activities. And third, the rigor cycle point out to ground \ac{DSR} on scientific knowledge, also known as knowledge base, for both design cycle activities, as well as to provide new knowledge to the knowledge base. All three cycles are by their nature iterative processes. Especially the design cycle needs to be run through a couple of times and the feedback of each iteration is taken into account for the next run. However, this process should determine at a certain point. Also the developed outcomes of this thesis -- the classification scheme and the qualitative as well as quantitative model -- are developed in an iterative approach. Nevertheless, the introduced outcomes below represent always the result of the final iteration. The interim results where discussed with the supervisors of this thesis, several experts, and a focus group.

%Without additional value
\begin{comment}
	\begin{table}[t]
		\caption[Design Science Research Guidelines]{Design Science Research Guidelines adapted from \citet[p. 83]{Hevner2007}}
		\label{tab:dsrg}
		\centering
		\begin{tabular}{L{.25\textwidth}L{.65\textwidth}}
				\toprule 
				\small \textbf{Guideline} &\small  \textbf{Description} \\ \midrule
				\small G1: Design as an Artifact & 
				\small Design-science research must produce a viable artifact in the form of a construct, a model, a method, or an instantiation.\\ \midrule
				\small G2: Problem Relevance&
				\small The objective of design-science research is to develop technology-based solutions to important and relevant business problems.\\ \midrule
				\small G3: Design Evaluation&
				\small The utility, quality, and efficacy of a design artifact must be rigorously demonstrated via well-executed evaluation methods.\\ \midrule
				\small G4: Research Contributions&
				\small Effective design-science research must provide clear and verifiable contributions in the areas of the design artifact, design foundations, and/or design methodologies.\\ \midrule
				\small G5: Research Rigor&
				\small Design-science research relies upon the application of rigorous methods in both the construction and evaluation of the design artifact.\\ \midrule
				\small G6: Design as a Search Process&
				\small The search for an effective artifact requires utilizing available means to reach desired ends while satisfying laws in the problem environment.\\ \midrule
				\small G7: Communication of Research&
				\small Design-science research must be presented effectively both to technology-oriented as well as management-oriented audiences.\\ \bottomrule
		\end{tabular}
	\end{table}
\end{comment}

\citet[pp. 52-56]{Peffers2007} developed a six-stage \ac{DSR} process model for the purpose of conducting \ac{DSR} in a stringent manner. The six activities -- (1) identify problem \& motivate, (2) define objectives of a solution, (3) design \& development, (4) demonstration, (5) evaluation, and (6) communication -- were applied as follows (cf. Figure \ref{fig:dsrm}):

\begin{enumerate}
	\item Identify problem \& motivate: As written above already, cloud computing represents an emerging field and especially the intermediary \ac{PaaS} layer is located at the so-called \textit{"Peak of Inflated Expectations"} \citep[p. 5]{Smith2012}. Moreover, a superior product -- the platform -- itself is nowadays not sufficient alone to operate a successful business. A \ac{PaaS} provider needs therefore also to consider the platform strategy as well as the platform ecosystem. Consequently, complex interdependencies are inherent in platform-based business models and these relationships are not always trivially observable. A model respectively artifact which is revealing the above mentioned problem is missing so that this research is considered as a problem-centered initiation. This first \ac{DSR} activity is discussed in Section \ref{ch:intro:mo}.
	\item Define objectives of a solution: The desired objective of this research is threefold. First, in order to classify \ac{PaaS} providers' business models and its main components a classification scheme is developed. Based on this classification scheme, the crucial interdependencies within \ac{PaaS} business models with regards to the platform adoption are investigated. This process results in the second, qualitative model, and third, quantitative model, outcome. The corresponding research questions and above mentioned objectives are discussed in Section \ref{ch:intro:rq}.
	\item Design \& development: Based on 23 explorative case studies the classification scheme for \ac{PaaS} business models was elaborated. For this process, the classification methodology introduced by \citet{Fettke2003} was used. Hereafter, the qualitative as well as quantitative model are developed, whereas the former aims to demonstrate the platform-based business model interdependencies and the latter provides the option to simulate the platform adoption process. Both models were developed using the concept of systems dynamics and follow the approach described by \citet{Sterman2000}. As previously mentioned, the classification scheme and both models were developed in an iterative approach. This third \ac{DSR} activity is discussed within the Chapters \ref{ch:sota}-\ref{ch:sfd}.
	\item Demonstration: All 23 investigated \ac{PaaS} providers' are classified by means of the elaborated classification scheme and thereby the usability of the classification scheme is demonstrated. In addition to this, the system dynamics models -- qualitative (\ac{CLD}) as well as quantitative (\ac{SFD}) -- are illustrated, both in textual and graphical form. This forth \ac{DSR} activity is discussed in Section \ref{ch:sota:cPaaS} as well as within the Chapters \ref{ch:cld} and \ref{ch:sfd}.
	\item Evaluation: As already mentioned, all three desired outcomes are developed and evaluated -- through a focus group and expert interviews -- in an iterative approach throughout the whole research process. The later introduced results represent always the version of the final iteration. As a \ac{PoC} the developed quantitative model is applied and simulated for \ac{PaaS} business models. This fifth \ac{DSR} activity is discussed in Section \ref{ch:sota:cm} as well as within the Chapters \ref{ch:cld}-\ref{ch:poc}.
	\item Communication: The document at hand represents the first publication of the conducted research. Moreover, it is planned to submit the results of this thesis as a conference paper either for the 47th \ac{HICSS} or for the 5th \ac{ICOSB}.
\end{enumerate}

The big picture of this thesis is provided in the following Figure \ref{fig:dsrm} and illustrates the \ac{DSR} methodology process according to \citet{Peffers2007} as well as the structure of the thesis at hand.

\begin{figure}[t]
	\centering
	% ****************************************************************************************************
% Design Science Research
% ****************************************************************************************************

\begin{tikzpicture}[scale=0.75, every node/.style={scale=0.75}, node distance = 4.5cm]

\node[draw,text width=8em,text centered, rectangle,rounded corners,minimum height=4em,thick,rotate=90] (a1) {
	\begin{minipage}{8em}\centering
		Identify\\ Problem\\ \& Motivate\\
		~\\
		\textit{Emerging \ac{PaaS} Market\\~\\Importance of strategy and ecosystems for platform-based business models.}
		~\\~\\~\\~\\~\\~\\~\\~\\
		(cf. Chapter \ref{ch:intro:mo})
		\\~
	\end{minipage}
};

\node[draw,text width=8em,text centered,rectangle,rounded corners,minimum height=4em,thick,rotate=90, right of=a1] (a2) {
\begin{minipage}{8em}\centering
		Define\\ Objectives of\\ a Solution\\
		~\\
		\textit{Classify PaaS providers' business models and its main components. Reveal crucial interdependencies within PaaS business models elements with regards to the platform adoption, using the concept of system dynamics.}
		~\\~\\
		(cf. Chapter \ref{ch:intro:ro})
		\\~
	\end{minipage}
};

\node[draw,text width=8em,text centered,rectangle,rounded corners,minimum height=4em,thick,rotate=90, right of=a2] (a3) {
\begin{minipage}{8em}\centering
		Design \&\\ Development\\
		~\\~\\
		\textit{Elaboration of a classification scheme for \ac{PaaS} business models. Development of a system dynamics model to demonstrate the adoption of platform-based business models.}
		~\\~\\~\\~\\~\\
		(cf. Chapter \ref{ch:sota} and \ref{ch:sd})
	\end{minipage}
};

\node[draw,text width=8em,text centered,rectangle,rounded corners,minimum height=4em,thick,rotate=90, right of=a3,] (a4) {
\begin{minipage}{8em}\centering
		Demonstration\\
		~\\~\\~\\
		\textit{By classifying several \ac{PaaS} business models the usability of the classification scheme is demonstrated. Through modeling actual\ldots}
		~\\~\\~\\~\\~\\~\\~\\
		(cf. Chapter \ref{ch:e})
		\\~
	\end{minipage}
};

\node[draw,text width=8em,text centered,rectangle,rounded corners,minimum height=4em,thick,rotate=90, right of=a4] (a5) {
\begin{minipage}{8em}\centering
		Evaluation\\
		~\\~\\~\\
		\textit{The developed classification scheme as well as the system dynamics model is evaluated in an interactive approach trough a focus group and expert interviews.}
		~\\~\\~\\~\\~\\~\\
		(cf. Chapter \ref{ch:e})
		\\~
	\end{minipage}
};

\node[draw,text width=8em,text centered,rectangle,rounded corners,minimum height=4em,thick,rotate=90, right of=a5] (a6) {
\begin{minipage}{8em}\centering
		Communication\\
		~\\~\\~\\
		\textit{As of writing this paper at hand, it is planned to write and submit a conference paper for the 47th \ac{HICSS}.}
		~\\~\\~\\~\\~\\~\\
	\end{minipage}
};

\node[draw,text width=5em,text centered,circle,rounded corners,minimum height=4em,thick,rotate=90, below of=a1,node distance=8cm] (e) {Problem-Centered Initiation};

\node[rotate=90, above left=3.51em and 1cm of a6] (ao6) {};
\node[rotate=90, above left=3.51em and 1cm of a5] (ao5) {};
\node[rotate=90, above left=3.51em and 1cm of a3] (ao3) {};
\node[rotate=90, above left=3.51em and 1cm of a2] (ao2) {};

%(a4.east)+(-1,0)

\path 
	(a1.70) edge [->,thick,>=stealth'] node [rotate=180,above left=0.2cm and 0cm of a1]{Inference} (a2.110)
	(a2.70) edge [->,thick,>=stealth'] node [rotate=180,above left=0.26cm and 0cm of a2]{Theory} (a3.110)
	(a3.70) edge [->,thick,>=stealth'] node [rotate=180,above left=0.26cm and 0cm of a3]{How to Knowledge} (a4.110)
	(a4.70) edge [->,thick,>=stealth'] node [rotate=180,above left=0.26cm and 0cm of a4]{Metrics, Analysis Knowledge} (a5.110)
	(a5.70) edge [->,thick,>=stealth'] node [rotate=180,above left=0.26cm and 0cm of a5]{Disciplinary Knowledge} (a6.110)
	(e.north) edge [->,thick,>=stealth'] (a1.south)
	
	(a6.north) edge [-,thick,>=stealth'] (ao6.south)
	(a5.north) edge [-,thick,>=stealth'] (ao5.south)
	(a3.north) edge [-,thick,>=stealth'] (ao3.south)
	(ao6.south) edge [-,thick,>=stealth'] node [rotate=90, above left=3.18em and 0.1cm of a4]{Process Iteration} (ao2.south)
	(ao3.south) edge [->,thick,>=stealth'] (a3.north)
	(ao2.south) edge [->,thick,>=stealth'] (a2.north);

\end{tikzpicture}


	\caption[Design Science Research Methodology -- Process Model]{Design Science Research Methodology -- Process Model adapted from \citet[p. 54]{Peffers2007}}
	\label{fig:dsrm}
\end{figure}