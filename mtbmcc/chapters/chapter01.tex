% ****************************************************************************************************
\chapter{Introduction}\label{ch:intro}
% ****************************************************************************************************

\section{Problem and Motivation}\label{ch:intro:mo}

\begin{itemize}
	\item "Understand the key elements and mechanisms in a specific business domain, as well as their relationships (Osterwalder \& Pigneur, 2002)", \citep[p. 303]{Pateli2004}
	\item "Focuses on describing the elements and relationships that outline how a company creates and markets value", \citep[p. 7]{Osterwalder2005}
	\item "Modern business models are increasingly complex, particularly those with strong ICT and ebusiness components. The relationship between the different elements of a business model and the decisive success factors are not always immediately observable. Therefore the process of modelling social systems and, in this case, business models help identify and understand the relevant elements in a specific domain and the relationships among them [Morecroft 1994; Ushold and King 1995]. In addition, the visual representation of a business model usually enhances understanding.", \citep[p. 14]{Osterwalder2005}
	\item "As simple as this framework may seem, its power lies in the complex interdependencies of its parts. Major changes to any of these four elements affect the others and the whole. Successful businesses devise a more or less stable system in which these elements bond to one another in consistent and complementary ways.", \citep[p. 53]{Johnson2008}
	\item "A useful way to represent business models is by means of a causal loop diagram, where choices and consequences are linked by arrows based on causality theories" \citep[p. 198]{Casadesus-Masanell2010}
\end{itemize}


\section{Research Questions and Objectives}\label{ch:intro:ro}

Cloud computing represents one of the emerging domains in the information technology field. The \ac{PaaS} layer provides new possibilities especially how software is developed and deployed. Nevertheless, almost all \ac{PaaS} offers are characterized by two- or even multi-sided business models. Simplified, \ac{PaaS} providers' address the two disjunct customer segments service providers and service consumers. Flourishing platforms need to address both customer segments successful. This aspect is also often referred to as platform ecosystem respectively platform strategies. \textit{"Who wins and who loses \ldots [the competition among \ac{PaaS} providers'] is not simply a matter of who has the best technology or the first product. It is often who has the best platform strategy and the best ecosystem to back it up"} \citep[p. 34]{Cusumano2010}. Within this thesis it is researched how \ac{PaaS} platforms are adopted and what are high leverage policies in the platform adoption process. This aim leads to the following Main Research Question:

\begin{MRQ}\label{mrq}
How to design Platform as a Service business models in order to achieve a high adoption rate?
\end{MRQ}

In order to answer the Main Research Question, three Sub Research Questions have been derived which are more precisely and focus on a narrower domain:

\begin{SRQ}\label{srq1}
How can business models of Platform as a Service providers' be classified?
\end{SRQ}
%-> Classification scheme

\begin{SRQ}\label{srq2}
How are the identified Platform as a Service business model elements -- especially in regards to the platform adoption -- interrelated?
\end{SRQ}
%-> Causal loop diagram

\begin{SRQ}\label{srq3}
How can the revealed relationships between Platform as a Service business model elements be quantified and the platform adoption simulated?
\end{SRQ}
%-> Stock and flow diagram

The current \ac{PaaS} market is distinguished by its immature nature and a remarkable provider fluctuation. As a first step, it is therefore inevitable to assess the current state of \ac{PaaS} providers' business models by use of explorative case studies. The information gained through the case studies is used to identify key elements within these business models and develop a classification scheme for \ac{PaaS} providers. This classification scheme represents the desired outcome of the \thref{srq1}. 

Based on the result of \thref{srq1}, the relationships between the identified classification criteria respectively characteristics are investigated. The focus during this process is how these interrelationships affect the platform adoption. A qualitative model -- also known as \ac{CLD} -- is designed to (graphically) feature these interrelations and constitute the answer for the \thref{srq2}.

As a last step, the qualitative model is transformed into a quantitative model -- also known as \ac{SFD}. By means of this model, the platform adoption process as well as the impact of single variables for the overall model can be simulated. The quantitative model represents the answer for the \thref{srq3} and also enables, in combination with the outcomes of the \thref{srq1} and \thref{srq2}, to provide valuable information for various \ac{PaaS} business model design options (Main Research Question).
	
\section{Research Methodology and Structure}\label{ch:intro:met}

The thesis at hand follows a \ac{DSR} methodology (\citealp{March1995}; \citealp{Hevner2004}; \citealp{Hevner2007}; and \citealp{Peffers2007}). According to \citep[p. 77]{Hevner2004}, \textit{"[\ac{DSR}] creates and evaluates IT artifacts intended to solve identified organizational problems."} Whereas unsolved problems addressed in unique or innovative ways and improved artifacts are developed for solved problems \citep[p. 81]{Hevner2004}. The overall goal of \ac{DSR} is its utility \citep[p. 80]{Hevner2004}.

\citet[pp. 255-262]{March1995} defined a two dimensional framework for research in IT. The first dimension describes four different research activities: build, evaluate, theorize, and justify. Possible outcomes are characterized in the second dimension: representational constructs, models, methods, and instantiations. This framework was further developed by \citet[pp. 78-81]{Hevner2004} and seven corresponding \ac{DSR} guidelines were introduced \citep[pp. 82-90]{Hevner2004} which are depicted in Table \ref{tab:dsrg}. How these guidelines are addressed in the course of this thesis is described further below.

\begin{table}[t]
	\centering
	\begin{tabular}{L{.2\textwidth}L{.7\textwidth}}
			\toprule 
			Guideline & Description \\ \midrule
			G1: Design as an Artifact & 
			Design-science research must produce a viable artifact in the form of a construct, a model, a method, or an instantiation.\\ \midrule
			G2: Problem Relevance&
			The objective of design-science research is to develop technology-based solutions to important and relevant business problems.\\ \midrule
			G3: Design Evaluation&
			The utility, quality, and efficacy of a design artifact must be rigorously demonstrated via well-executed evaluation methods.\\ \midrule
			G4: Research Contributions&
			Effective design-science research must provide clear and verifiable contributions in the areas of the design artifact, design foundations, and/or design methodologies.\\ \midrule
			G5: Research Rigor&
			Design-science research relies upon the application of rigorous methods in both the construction and evaluation of the design artifact.\\ \midrule
			G6: Design as a Search Process&
			The search for an effective artifact requires utilizing available means to reach desired ends while satisfying laws in the problem environment.\\ \midrule
			G7: Communication of Research&
			Design-science research must be presented effectively both to technology-oriented as well as management-oriented audiences.\\ \bottomrule
	\end{tabular}
	\caption[Design Science Research Guidelines]{Design Science Research Guidelines adapted from \citet[p. 83]{Hevner2007}}
	\label{tab:dsrg}
\end{table}

\citet[pp. 52-56]{Peffers2007} developed a six-stage \ac{DSR} process model for the purpose of conducting \ac{DSR} in a stringent manner. The six activities -- (1)Problem Identification \& Motivation, (2) Define Objectives of a Solution, (3) Design \& Development, (4) Demonstration, (5) Evaluation, and (6) Communication -- were applied as follows (cf. Figure \ref{fig:dsrm}):

\begin{enumerate}
	\item Problem Identification \& Motivation:
	\item Define Objectives of a Solution:
	\item Design \& Development:
	\item Demonstration:
	\item Evaluation:
	\item Communication:
\end{enumerate}



\begin{figure}[tb]
	\centering
	% ****************************************************************************************************
% Design Science Research
% ****************************************************************************************************

\begin{tikzpicture}[scale=0.75, every node/.style={scale=0.75}, node distance = 4.5cm]

\node[draw,text width=8em,text centered, rectangle,rounded corners,minimum height=4em,thick,rotate=90] (a1) {
	\begin{minipage}{8em}\centering
		Identify\\ Problem\\ \& Motivate\\
		~\\
		\textit{Emerging \ac{PaaS} market\\~\\Importance of platforms strategy and ecosystems for platform-based business models.\\~\\Complex interdependencies which are not always trivially observable.}
		~\\~\\~\\~\\~\\~\\
		(cf. Section \ref{ch:intro:mo})
		\\~\\
	\end{minipage}
};

\node[draw,text width=8em,text centered,rectangle,rounded corners,minimum height=4em,thick,rotate=90, right of=a1] (a2) {
\begin{minipage}{8em}\centering
		Define\\ Objectives of\\ a Solution\\
		~\\
		\textit{Classify PaaS providers' business models and its main components.\\~\\ Reveal crucial interdependencies within PaaS business models elements with regards to the platform adoption, using the concept of system dynamics.}
		~\\~\\~\\~\\~\\
		(cf. Section \ref{ch:intro:ro})
		\\~\\
	\end{minipage}
};

\node[draw,text width=8em,text centered,rectangle,rounded corners,minimum height=4em,thick,rotate=90, right of=a2] (a3) {
\begin{minipage}{8em}\centering
		Design \&\\ Development\\
		~\\~\\
		\textit{Elaboration of a classification scheme for \ac{PaaS} business models based on explorative case studies.\\~\\ Development of a system dynamics model (qualitative and quantitative) to demonstrate and simulate the adoption of platform-based business models.}
		~\\~\\~\\
		(cf. Chapter \ref{ch:sota}, \ref{ch:cld}, and \ref{ch:sfd})
	\end{minipage}
};

\node[draw,text width=8em,text centered,rectangle,rounded corners,minimum height=4em,thick,rotate=90, right of=a3,] (a4) {
\begin{minipage}{8em}\centering
		Demonstration\\
		~\\~\\~\\
		\textit{By classifying several \ac{PaaS} business models the usability of the classification scheme is demonstrated.\\~\\ The developed system dynamics model is applied and simulated for several \ac{PaaS} business models as \ac{PoC}.}
		~\\~\\~\\~\\
		(cf. Section \ref{ch:sota:cPaaS} and Chapter \ref{ch:poc})
	\end{minipage}
};

\node[draw,text width=8em,text centered,rectangle,rounded corners,minimum height=4em,thick,rotate=90, right of=a4] (a5) {
\begin{minipage}{8em}\centering
		Evaluation\\
		~\\~\\~\\
		\textit{The developed classification scheme as well as the model is evaluated in an iterative approach trough a focus group and expert interviews.}
		~\\~\\~\\~\\~\\~\\~\\~\\~\\~\\~\\
		(cf. Section \ref{ch:sota:cm} and Chapter \ref{ch:cld}-\ref{ch:poc})
	\end{minipage}
};

\node[draw,text width=8em,text centered,rectangle,rounded corners,minimum height=4em,thick,rotate=90, right of=a5] (a6) {
\begin{minipage}{8em}\centering
		Communication\\
		~\\~\\~\\
		\textit{As of writing this paper at hand, it is planned to write and submit a conference paper, either for the 47th \ac{HICSS} or for the 5th \ac{ICOSB}.}
		~\\~\\~\\~\\~\\
	\end{minipage}
};

\node[draw,text width=5em,text centered,circle,rounded corners,minimum height=4em,thick,rotate=90, below of=a1,node distance=9cm] (e) {Problem-Centered Initiation};

\node[rotate=90, above left=3.51em and 1cm of a6] (ao6) {};
\node[rotate=90, above left=3.51em and 1cm of a5] (ao5) {};
\node[rotate=90, above left=3.51em and 1cm of a3] (ao3) {};
\node[rotate=90, above left=3.51em and 1cm of a2] (ao2) {};

%(a4.east)+(-1,0)

\path 
	(a1.70) edge [->,thick,>=stealth'] node [rotate=180,above left=0.2cm and 0cm of a1]{Inference} (a2.110)
	(a2.70) edge [->,thick,>=stealth'] node [rotate=180,above left=0.26cm and 0cm of a2]{Theory} (a3.110)
	(a3.70) edge [->,thick,>=stealth'] node [rotate=180,above left=0.26cm and 0cm of a3]{How to Knowledge} (a4.110)
	(a4.70) edge [->,thick,>=stealth'] node [rotate=180,above left=0.26cm and 0cm of a4]{Metrics, Analysis Knowledge} (a5.110)
	(a5.70) edge [->,thick,>=stealth'] node [rotate=180,above left=0.26cm and 0cm of a5]{Disciplinary Knowledge} (a6.110)
	(e.north) edge [->,thick,>=stealth'] (a1.south)
	
	(a6.north) edge [-,thick,>=stealth'] (ao6.south)
	(a5.north) edge [-,thick,>=stealth'] (ao5.south)
	(a3.north) edge [-,thick,>=stealth'] (ao3.south)
	(ao6.south) edge [-,thick,>=stealth'] node [rotate=90, above left=3.18em and 0.1cm of a4]{Process Iteration} (ao2.south)
	(ao3.south) edge [->,thick,>=stealth'] (a3.north)
	(ao2.south) edge [->,thick,>=stealth'] (a2.north);

\end{tikzpicture}


	\caption[Design Science Research Methodology -- Process Model]{Design Science Research Methodology -- Process Model adapted from \citet[p. 54]{Peffers2007}}
	\label{fig:dsrm}
\end{figure}

