% ****************************************************************************************************
\chapter{State--of--the--Art PaaS Providers' Business Models}\label{ch:sota}
% ****************************************************************************************************


\section{Business Model Conceptualization/ Reference Model}

\hangbibentry{Johnson2008}
\hangbibentry{Kagermann2011}
\hangbibentry{Osterwalder2005}

\section{The Case of CloudBees}

CloudBees was founded in the beginning of 2012 and introduced its identically named platform as a service offer. Many of CloudBees' management team members, including the founder, worked prior for companies like Oracle, Sun, IBM, and VMware or were involved in development projects around JBoss and Jenkins CI. Already the background of CloudBees' management reveals the main objective -- a platform as a service solution truly dedicated around the Java programming language ecosystem, supporting the whole application lifecycle \citep{CloudBees2013}.

The CloudBees platform is composed of two distinct but related concepts. First, all development, integration, and testing tasks are performed within the DEV@cloud. Core of this part of the platform is the Jenkins Continuous Integration (CI) server. Once an application is ready to go live, the application is migrated to the RUN@cloud. Basically, the RUN@cloud represents a traditional application server, including the functionalities load balancing, scalability, and high availability, based on various cloud infrastructures.

Mainly two different customer groups are attracted by CloudBees platform solution -- business end-consumer and independent software vendors. The first-mentioned group utilizes the platform to manage the entire application lifecycle -- including development, quality assurance and testing, deployment and release management, as well as monitoring and administration. Due to the fact, that all applications upon the CloudBees platform are based on standard Java and not on specific CloudBees APIs, potential customers are not faced with a so called lock-in effect. Moreover, platform customers can select between multi-tenant and dedicated RUN@cloud's. In case of dedicated instances, they can choose the appropriate cloud infrastructure -- private, public, and hybrid provisioning scenarios are supported. Via the CloudBees Partner Ecosystem, business end-consumer can purchase third-party applications and services and extend the CloudBees platform functionalities. 

Platform extensions are provided by the second customer group, the independent software vendors. CloudBees uniquely defined mission -- support the whole application lifecycle for standard Java and Java Virtual Machine (JVM) based applications within the cloud -- enables independent software providers to expand the already existing platform capabilities by deployment and testing tools, comprehensive monitoring features, as well as different databases. All third-party applications, tools, and services are listed within the CloudBees Partner Ecosystem. Especially for small independent software providers, but certainly also for all other independent software providers, it is beneficial, that the payment handling, inclusive billing, for third-party extensions is performed by CloudBees.

The overall revenue of CloudBees is composed of two revenue streams -- through CloudBees' partners and CloudBees' platform customers. Consulting partners, i.e. system integrators, need to pay an annual partner fee -- silver partners USD 2.000 and gold partners USD 5.000 -- to become a certified CloudBees Service Partner. For all third-party applications, tools, and services provided by independent software providers, denoted by CloudBees as Technology Partners, CloudBees gets a default revenue share of 30\%. Nevertheless, CloudBees offers the possibility to negotiate the revenue share rate individually.

Business end-consumer pay for the CloudBees platform a combination of subscription and transaction based fees. The DEV@cloud is available as four different packages -- Free USD 0, Base USD 15, Pro USD 50, and Enterprise USD 100, all prices per month -- including different features and quotas. A different number of build minutes is included within all packages and additional build minutes can be purchased for USD 0.106 respectively 0.425 per hour. A multi-tenant RUN@cloud is either free (no additional features or services can be purchased) or charged per hour, based on the select application server and SSL connections. As mentioned above, the dedicated RUN@cloud can basically be hosted at any cloud infrastructure, with the result that a vast number of pricing models is possible. For instance, the CloudBees RUN@cloud platform hosted at Amazon Web Service infrastructure is priced between USD 153 - 644 per instance monthly. Furthermore, CloudBees is offering also a MySQL database which is priced based on the selected RUN@cloud \citep{CloudBees2013}.

%insert BM CLoudBees

\section{The Case of GigaSpaces Cloudify}

Within the evolving platform as a service ecosystem are just a few open source platform solutions available -- Cloudify is one of them. Since the beginning of 2012, the Cloudify Open PaaS Stack is available on the market. Cloudify itself and the Cloudify Community are promoted by the company GigaSpaces \citep{GigaSpaces2013a}.

The basic concept of Cloudify is the so-called external blueprint, also named as recipe. Within a blueprint the deployment and post-deployment activities of an application are described. This concept enables developers to migrate existing applications to cloud infrastructure without the need to adopt or change the source code. Moreover, based on the blueprint concept, applications can be deployed at or migrate between private, public, or even hybrid cloud infrastructures \citep{GigaSpaces2013a}.

Two distinct customer groups can use the Cloudify platform offer: (1) business end-consumers who want to move existing non-cloud applications to the cloud or develop and operate cloud applications and (2) independent software vendors who want to offer migration services (migrate on-premise applications onto cloud infrastructure). The value proposition for these two stakeholder groups include the promise to move any application without code or application architecture changes to the cloud, develop the external blueprint for an application once and be able to deploy the application at different cloud infrastructures, as well as full control over the Cloudify Open PaaS Stack. Furthermore, an interactive shell and web-based user-interface for monitoring purposes are provided \citep{GigaSpaces2013a}.

As usual for many open source projects, a flourishing community -- here the Cloudify Community -- is actively participating around the open source product. Within the Cloudify Community product related knowledge is shared via support forums, documentations, events, blogs, and videos. Moreover, the Cloudify platform source code and sample Cloudify blueprints respectively recipes are public available via GitHub \citep{GigaSpaces2013b,GitHub2013,GitHub2013a}.

GigaSpaces' Cloudify platform is offered as an open source platform and thereby free of charge for the platform software itself. However, it should be noted that Cloudify platform users still need the corresponding cloud infrastructure, for instance offered for a fee by Amazon Web Services, Rackspace, and HPCloud. Various charged services with regards to Cloudify are provided by GigaSpaces, including technical support, trainings, and consultancy services \citep{GigaSpaces2013a}.

%insert BM GigaSpaces Cloudify

\section{The Case of Facebook Developers}

Facebook -- the social network with more than a billion monthly active users, 618 million daily active users, and 680 million monthly active users using mobile devices to access Facebook's social network; all figures as of December 2013 -- introduced its platform as a service offer, called Facebook Developers, in May 2007 \citep{Facebook2013}. By the end of March 2012, more than nine million applications and websites are built upon or utilizing features of the Facebook Developers platform and are integrated with Facebook \citep{Facebook2013}. Facebook Developers gains its attractiveness especially through Facebook's huge user base and thereby millions of potential users of applications and websites.

The Facebook Developers platform can be used to integrate Facebook features and services into (mobile) applications as well as websites. In the other direction the platform can be used to integrate (mobile) applications into Facebook. Furthermore, advertising providers provide advertisement services which can be used by developers to monetize their efforts. The Facebook Developers platform provides a comprehensive set of tools and features to support different kinds of development and administration tasks \citep{Facebook2013a}: Facebook software development kits (SDK) for iOS, Android, JavaScript, and PHP; third-party software development kits (SDK) for .NET (C\#), Falsh (ActionScript), Python, Java (Spring), Java (BlackBerry), Ruby, and Node.js; Facebook application programming interfaces (API) like Login, Graph API, and Facebook Query Language (FQL); and tools like the Graph API Explorer as well as a JavaScript Test Console. Via social plugins, (mobile) applications and websites can integrate Facebook features and services, for instance the Like, Send, Follow, and Login Button; the Comments Box; the Activity Feed; as well as the Recommendation Box and Bar. Applications are distributed over the Facebook App Center (marketplace) with linkage functionality for native mobile applications -- for iOS applications linkage to the App Store and for Android applications linkage to the Google Play Store -- as well as on-demand provisioning for non-native mobile or browser-based applications.

Facebook is generating monetary revenue mainly with the following three revenue streams. First, new or already existing applications can be promoted through a fee-based promoting self-service. Facebook offers the possibility to define the target audience -- for instance region, age, and gender --, the promoting goal -- i.e. get installs --, and the promoting budget. Second, application users can purchase digital and virtual goods in social applications based on the currency Facebook Credits, since end of 2012 also the option local currency pricing is possible \citep{Facebook2013a}. In order to use in-app payment, users need to buy Facebook Credits or, in case of local currency pricing, pay directly with their credit card, PayPal, or another payment method. The application or service provider on the other side can change the Facebook Credits obtained through their applications and services into real money. Facebook takes a 30\% charge for this service. And third, most likely the most important revenue stream is generated through personalized advertisements, for instance within applications and websites.

Furthermore, Facebook generates also non-monetary revenue in respect of even more detailed and valuable user data capturing. It is possible to login to various websites and services (build with the help of features of the Facebook Developer platform) with a valid Facebook account -- a kind of single sign-on (SSO) system which can be considered as a lock-in effect and allows Facebook to collect valuable user data from various external sources.

%insert BM Facebook Developers

\section{The Case SAP NetWeaver Cloud}

SAP, a German enterprise software vendor with more than 200.000 customers worldwide, launched a new platform as a service solution -- named SAP NetWeaver Cloud -- at the end of 2012 \citep{SAP2013a, SAP2013b}. SAP NetWeaver Cloud aims to support the development, deployment, and management of standalone as well as integrated cloud applications. The platform itself and the applications build upon the platform are both hosted by SAP with a guaranteed system availability of 99.9\% \citep{SAP2013b}.

SAP's new platform as a service solution is targeting four diverse stakeholder groups. Application customers who want to use pure software as a service applications as well as complement existing SAP and non-SAP on-premise systems with cloud applications, build the first customer group. Presumably the most important part of SAP's offer for this stakeholder group, is the SAP Store with certified third-party applications (i.e. quality assurance) and the guaranteed system availability of 99.9\%.

Development partners -- the second stakeholder group -- utilize the platform to develop, deploy, manage, and market business software as a service applications globally. SAP's existing installed base respectively the more than 200.000 customers worldwide represent potential customers for business applications build upon SAP NetWeaver Cloud. Due to the fact that the applications are built with the programming language Java, SAP is providing an Eclipse plugin as well as a corresponding software development kit. Hereby, developers are able to develop applications within the well-known and open integrated development environment Eclipse as well as test and debug applications locally. Furthermore, various integration capabilities to SAP and non-SAP on-premise systems are provided respectively currently under development. On top of that, all applications can use SAP HANA -- SAP's in-memory database technology. Further offerings for development partners include the SAP Store as channel, comprehensive support services, and best practice sharing in regards to application pricing.

The third stakeholder group are the so-called platform customers. These customers subscribe and use the SAP NetWeaver Cloud platform to be able to react continuously to internal and market demands in a cost efficiency manner. In most cases, these customers maintain their own internal IT department, which develops business cloud applications integrable to SAP and non-SAP on premise systems. The offer for this group include the Eclipse plugin and software development kit, the integration capabilities, access to SAP HANA, the SAP Store to purchase third-party applications, support services, as well as the service level agreement (SLA) concerning the system availability.

Finally, SAP offers an Individual Developer License to the fourth stakeholder group -- the individual developers. With this free, however non-productive, license developers get access to the SAP NetWeaver Cloud platform and can experience as well as test the platform capabilities. Obvious, the offering for individual developers include the Eclipse plugin and corresponding software development kit as well as various support services, like tutorials and sample applications, all with the objective to keep the entry barriers as low as possible. SAP aims that these individual developers feel confident with the SAP NetWeaver Cloud platform capabilities and decide to participate as development partners in the ecosystem in the near future.

The revenue for SAP is based on the following revenue streams: As mentioned above, the SAP NetWeaver Cloud platform itself can be subscribed by customers as a whole. Most likely these customers are upper small and medium-sized enterprises (SMEs) or large enterprises. The platform can be subscribed as a pre-defined package, including certain quotas for structured and unstructured storage, bandwidth, connections to SAP and non-SAP on-premise systems, as well as compute units. Currently, three different packages are available \citep{SAP2013b}: (1) Starter Package EUR 370 per month, (2) Professional Package EUR 4.800 per month, and (3) Premium Package EUR 16.000 per month. Furthermore, customers can also build their own package, based on the following six metrics: (1) virtual machines, (2) unstructured storage (e.g. documents), (3) structured storage (HANA), (4) structured storage (RDMS), (5) bandwidth, and (6) connections to third party systems (SAP systems for free). 

Independent software vendors, startup companies, and solution providers -- summarized as development partners -- provide software as a service applications upon the SAP NetWeaver Cloud platform \citep{SAP2013a}. In order to utilize SAP's platform, development partners need to pay an annual partner fee of EUR 1.990. All third-party applications available within the SAP store are certified by SAP. Application providers need to pay an initial application certification fee (EUR 990) and recurring annual fees (EUR 495). Furthermore, SAP gets a 15\% revenue share (offset against the annual partner fee) for all applications based on SAP NetWeaver Cloud and sold via the SAP Store.


%insert BM SAP NetWeaver Cloud

\section{Classification Approach}

\hangbibentry{Fettke2003}