% ****************************************************************************************************
\chapter{Conclusion}\label{ch:dc}
% ****************************************************************************************************

	% ****************************************************************************************************
	\section{Summary}\label{ch:dc:s}
	% ****************************************************************************************************

	The goal of this thesis was to facilitate a better understanding of the dynamics of \acf{PaaS} business models and thus to answer the main research question -- \textit{How to design Platform as a Service business models in order to achieve a high adoption rate?} -- as well as its three sub-research questions. In order to address this research project, a \acf{DSR} approach was chosen, resulting in three artifacts, each one based on the previous one and represent the solutions to the three sub-research questions. 
	
	First, a classification scheme presenting the main design elements of \ac{PaaS} business models was created in order to answer the first sub-research question: \textit{How can Platform as a Service business models be classified?} In order to assess the current state of \ac{PaaS} provider's business models and identify their key elements, the current business models of 23 \ac{PaaS} providers were analyzed. The selection of the \ac{PaaS} providers analyzed in this thesis was based on technical reports and market reports which were whittled down to receive a manageable set of representative \ac{PaaS} providers. The classification scheme for \ac{PaaS} business models which finally resulted consists of five essential classification criteria and their characteristics, viz. (1) customer segment: \ac{IT} startup, \acf{SI}, \acf{ISV}, platform customer, application customer; (2) core value proposition: distribution channel, application-based integration, development, integration; (3) governance model: strictly limited, partly limited, open; (4) technical scope: limited technical capabilities, extensive technical capabilities; and (5) revenue stream: subscription, transaction, revenue sharing, additional services, \acf{OSS}.
	
	Second, a qualitative system dynamics model (\acf{CLD}), illustrating the complex dynamics of \ac{PaaS} business models was developed with a view to the second sub-research question: \textit{How are the different elements of a Platform as a Service business model interrelated, especially with regard to the platform adoption?} Business models, in particular those with several customer segments and crucial network effects, are a complex field of study. Their inherent interrelationships and interdependencies are not always immediately observable. The impact of these relationships on the rate of platform adoption is of especially great importance. Based on the outcome of the first sub-research question, as well as using the information and insights gained from the 23 explorative case studies, the essential interdependencies, i.e. high-leverage interventions and policies, between the previously mentioned classification criteria and characteristics were revealed. On the one hand, the \acf{CVP} mainly influences the adoption rate and is itself affect by the classification criteria core value proposition, governance model, and technical scope. Moreover, several business model elements were consolidated and also influence the customer value proposition, viz. additional services, platform improvements, and customer base. On the other hand, interdependencies between the five different customer segments have been revealed. Both, the core concept of the customer value proposition and the customer segment interdependencies, resulted in the identification of feedback loops, in this case especially self-reinforcing (positive) feedback loops have been found.
	
	Third, the qualitative model was translated into a quantitative model (\acf{SFD}), providing the opportunity to simulate the adoption of \ac{PaaS} business models, which answers the third sub-research question: \textit{How can the relationships between Platform as a Service business model elements which have been revealed be quantified, and how can the platform adoption be simulated?} The two central concepts of system dynamics theory, i.e. of \acp{SFD}, are stocks and flows, where stocks accumulate flows over time. By applying this concept to the developed \ac{CLD}, 11 crucial stocks in the domain of \ac{PaaS} business models were revealed. The five different customer segments (cf. classification scheme) are modeled each with a potential customers stock as well as customer population stock. For the purpose of measuring the overall customer base a same-named stock is used. The customer segment stocks increase and decrease through the respective customer segment adoption rate. In order to determine these adoption rates, both, the earlier identified key elements of \ac{PaaS} business models, i.e. classification scheme, and the direct as well as indirect networks effects identified in the \ac{CLD} are utilized.
	
	These artifacts taken together allow a systematic review of \ac{PaaS} business models with special regard to the adoption dynamics and facilitate the business model design process through analysis as well as simulation features. 

	

	% ****************************************************************************************************
	\section{Discussion}\label{ch:dc:d}
	% ****************************************************************************************************
		
	The results presented in the thesis at hand constitute a significant scientific and practical contribution. The three artifacts, i.e. the classification scheme as well as the qualitative and quantitative system dynamics models, contribute to the knowledge base and facilitate a better understanding of \ac{PaaS} business models. In particular, the \ac{CLD} explaining the inherent dynamics of \ac{PaaS} business models enhances existing knowledge about \ac{PaaS} business models which is mainly structural in kind. While other authors such as \citet{BenLagha2001}, \citet{Klueber2000}, and \citet{Kiani2009} focus on explaining the interrelationships of business model elements, and \citet{Cusumano2010}, \citet{Gawer2008}, \citet{Cusumano2002}, \citet{Eisenmann2006}, and \citet{Beimborn2011} focus on general platform strategies, this thesis takes both approaches into account on the basis of actual \ac{PaaS} business models. The system dynamics approach presented in the thesis at hand focuses on and explains the impact of the interdependencies of the previously identified \ac{PaaS} business model design elements (derived from the classification scheme) on the overall platform adoption. Another contribution of the research presented here is the analysis and explanation of the interdependencies among the five revealed target customer segments in multi-sided business models. Moreover, the quantitative model represents a major contribution of the thesis at hand, going one step further than other researchers so far, for instance \citet{Klueber2000} and \citet{Kiani2009}. It provides a first attempt to simulate the adoption of \ac{PaaS} business models. Besides providing general system dynamics models explaining the interdependencies within \ac{PaaS} business models and facilitating the simulation of such business models, a \acf{PoC} study examining the platform adoption was performed. The applicability of system dynamics to the analysis of multi-sided \ac{PaaS} business models was illustrated in a concrete example and generalized for practical use.

	The results presented also provide a contribution with practical relevance. The knowledge about typical balancing and self-reinforcing feedback loops presented in the generic \ac{CLD} can help \ac{PaaS} providers to understand the adoption dynamics of their platform and the factors influencing it. Moreover, utilizing the simulation model can support these providers beyond merely understanding these dynamics. Potential managerial decisions can be simulated, thus evaluating their long-term impact on the platform adoption. Based on the generic feedback loops, the example of an evaluation clearly demonstrates how potentially weak features of existing business models can be identified.
	
	All three artifacts were developed using scientific methods in an iterative approach in line with \citet{Hevner2007} and \citet{Peffers2007}. The classification scheme was developed using the classification methodology introduced by \citet{Fettke2003} and is based on 23 explorative case studies, in accordance with \citet{Eisenhardt1989} and \citet{Yin2008}. Both system dynamics models, i.e. qualitative and quantitative, were developed using the system dynamics theory in line with \citet{Sterman2000,Sterman2001}. These system dynamics models are built upon the concept of an adoption cycle \citep{Sterman2001} and highlight two main aspects crucial to the adoption of \ac{PaaS} business models. First, each customer segment addressed needs a specific and unique \ac{CVP}, based on its individual needs. This aspect is also emphasized by \citet{Johnson2008} in his business model conceptualization which was used for the 23 case studies. Second, the interdependencies between the target customer segments, i.e. cross-sided network effects, are decisive for the overall platform adoption process.

	% ****************************************************************************************************
	\section{Limitations and Further Research}\label{ch:dc:lfw}
	% ****************************************************************************************************

	Despite the promising results, the study has limitations that might affect the generalizability of the results and that need to be mentioned. Even though the classification scheme was evaluated in 23 case studies, the system dynamics models were evaluated only in a few \ac{PoC} studies. This number is sufficient to ensure rigor and relevancy, but might limit the generalizability of both system dynamics models.

	In this thesis the design elements as well as feedback structure, i.e. dynamics, of \ac{PaaS} business models have been exhibited and illustrated in the form of a classification scheme and system dynamics models. Further research might include an extensive evaluation of the system dynamics models, both \ac{CLD} and \ac{SFD}, using various \ac{PoC} studies. Moreover, research projects dealing with the mathematical foundations of the model, such as assumptions, variables, parameters, codomains, etc., could help to identify specific high-leverage interventions and policies in the domain of \ac{PaaS} business models. Another aspect which has not been considered in this thesis is the decreasing value of software. Platform modules, especially applications, built upon \ac{PaaS} platforms are rather lightweight solutions with a relatively short lifespan and thus will become less valuable over time. This fact needs to be taken into account and investigated further. Finally, in this first simulation model the impact of the various influences is equal for all customer segments. Further research could investigate the magnitude of these factors and define factors specific to each customer segment. Thus, the detailed simulation of \ac{PaaS} business models, based on the knowledge gained here, constitutes interesting further research.