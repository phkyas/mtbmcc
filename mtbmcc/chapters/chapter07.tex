% ****************************************************************************************************
\chapter{Discussion and Conclusions}\label{ch:dc}
% ****************************************************************************************************

%Model Boundary - Organization resp. single PaaS product (not the PaaS market as a whole)

 %The specific impact of the core value proposition onto the <customer segment> customer value proposition depends on the actual <customer segment> and needs to be further investigated for the transformation into a quantified model.

%Also here the specific impact of both variables onto the <customer segment> customer value proposition depends on the actual <customer segment> and needs to be further investigated for the transformation into a quantified model.

%Once again, the specific impact of the variable additional services onto the <customer segment> customer value proposition depends on the actual <customer segment> and needs to be further investigated for the transformation into a quantified model.

% As already mentioned above, the specific impact of the variable platform improvements onto the <customer segment> customer value proposition, additional services, and technical scope depends on the actual <customer segment> and needs to be further investigated for the transformation into a quantified model.

\begin{comment}

	\begin{itemize}
		\item vorerst keine CF (contact frequency) wie bei Sterman, kann im ersten versuch vernachlässig werden
		\item additiv oder multiplikativ
		\item annahme: total population = adopters + potentail adopters
		\item Flow von ITS zu ISV
		\item Tradeoff against the uncertainties inevitably inherent in quantification
		\item How much value does quantified modelling add to qualitative analysis? (Coyle 2000)
		\item How can we guard against models that risk becoming plausible nonsense? (Coyle 2000)
		\item damping factor/ decreasing value of software
		\item zwei modellierungsarten: abolut und relative
	\end{itemize}
	
\end{comment}

The goal of this thesis was to facilitate a better understanding of the dynamics of \acf{PaaS} business models and thus answering the main research question -- \textit{How to design Platform as a Service business models in order to achieve a high adoption rate?} -- as well as three sub-research questions. In order to address this research project, a \acf{DSR} approach was chosen, resulting in three artifacts which are based on each other and represent the outcomes of the three sub-research questions. First, a classification scheme presenting the main design elements of \ac{PaaS} business models was created in order to answer the first sub-research questions -- \textit{How can business models of Platform as a Service business models be classified?} Second, a qualitative system dynamics model -- \acf{CLD} --, illustrating the complex dynamics of \ac{PaaS} business models was developed based on the second sub-research question -- \textit{How are the identified Platform as a Service business model elements -- especially in regards to the platform adoption -- interrelated?} And third, the qualitative model was further transferred into a quantitative model -- \acf{SFD} -- providing the opportunity to simulate the adoption of \ac{PaaS} business models and answers the third sub-research question -- \textit{How can the revealed relationships between Platform as a Service business model elements be quantified and the platform adoption simulated?}

All three artifacts were developed using scientific methods in an iterative approach in line with \citet{Hevner2007}. The classification scheme was developed using the classification methodology introduced by \citet{Fettke2003} and is based on 23 explorative case studies according to \citet{Eisenhardt1989} and \citet{Yin2008}. Both system dynamics models -- qualitative and quantitative -- have been developed using the system dynamics theory in line with \citet{Sterman2000,Sterman2001}. These models are built upon the adoption cycle concept \citep{Sterman2001} and highlight two main aspects crucial for the adoption of \ac{PaaS} business models. First, each addressed customer segment needs a specific and dedicated \acf{CVP}, based on its individual needs. This aspect was also emphasized by \citet{Johnson2008} and included in his business model conceptualization which has been used for the 23 case studies. And second, the interrelationships between customer segments -- cross-sided network effects -- are decisive for the overall platform adoption.

The results presented in the thesis at hand provide a significant scientific and practical contribution. The three artifacts -- the classification scheme as well as the qualitative and quantitative system dynamics model -- contribute to the knowledge base and facilitate a better understanding of \ac{PaaS} business models. In particular, the \ac{CLD} explaining the inherent dynamics of \ac{PaaS} business models, enhances existing rather structural knowledge about \ac{PaaS} business models. While other authors as named in the related research section focused on explaining the interrelationships of business model elements, the system dynamics approach presented in the thesis at hand focuses on and explains the impact of interdependencies of the previously identified business model design elements (derived from the classification scheme) with respect to their adoption. Another contribution of the research presented here is the analysis and explanation of the interdependencies among the five revealed target customer segments in multi-sided business models. Moreover, the quantitative model represents major contribution of the thesis at hand, going one step further as other researchers so far and providing a first attempt to simulation the adoption of \ac{PaaS} business models. Besides providing general system dynamics models explaining the interdependencies within \ac{PaaS} business models as well as to facilitate the simulation of such business models examining the platform adoption, the applicability of system dynamics for the analysis of multi-sided \ac{PaaS} business models was illustrated on a concrete example and generalized for practical use.

The presented results provide also a contribution with practical relevance. The knowledge about typical balancing and self-reinforcing feedback loops presented in the generic \ac{CLD} can help \ac{PaaS} providers to understand the adoption dynamics of their platform and which factors have influence on it. Moreover, utilizing the simulation model can support these providers beyond just understanding these dynamics. Potential managerial decisions can be simulated thus evaluating their long run impact on the platform adoption. The evaluation example clearly demonstrates how based on the identified generic feedback loops, potential week features of existing business models can be identified.

Despite of the promising results, the study has limitations that might affect the generalizability of the results and that need to be mentioned. Even though the classification scheme has been evaluated in 23 case studies, the developed system dynamics models have been evaluated only in several \acf{PoC} studies. This number is sufficient to assure rigor and relevant results, but might limit the generalizability of both system dynamics models.

Within this thesis the design elements as well as feedback structure -- dynamics -- of \ac{PaaS} business models have been exhibited and illustrated in form of a classification scheme and system dynamics models. Further research might include an extensive evaluation of the system dynamics models -- \ac{CLD} and \ac{SFD} -- using various \ac{PoC} studies. Moreover, research projects with respect of the mathematical model foundations -- assumptions, variables, parameters, codomains, and the like -- could yield to a specific identification high-leverage interventions and policies in the domain of \ac{PaaS} business models. Thus, the detailed simulation of \ac{PaaS} business models based on the gained knowledge would be interesting further research.
