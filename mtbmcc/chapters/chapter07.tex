% ****************************************************************************************************
\chapter{Discussion and Conclusions}\label{ch:dc}
% ****************************************************************************************************

%Model Boundary - Organization resp. single PaaS product (not the PaaS market as a whole)

 %The specific impact of the core value proposition onto the <customer segment> customer value proposition depends on the actual <customer segment> and needs to be further investigated for the transformation into a quantified model.

%Also here the specific impact of both variables onto the <customer segment> customer value proposition depends on the actual <customer segment> and needs to be further investigated for the transformation into a quantified model.

%Once again, the specific impact of the variable additional services onto the <customer segment> customer value proposition depends on the actual <customer segment> and needs to be further investigated for the transformation into a quantified model.

% As already mentioned above, the specific impact of the variable platform improvements onto the <customer segment> customer value proposition, additional services, and technical scope depends on the actual <customer segment> and needs to be further investigated for the transformation into a quantified model.

\begin{comment}

	\begin{itemize}
		\item vorerst keine CF (contact frequency) wie bei Sterman, kann im ersten versuch vernachlässig werden
		\item additiv oder multiplikativ
		\item annahme: total population = adopters + potentail adopters
		\item Flow von ITS zu ISV
		\item Tradeoff against the uncertainties inevitably inherent in quantification
		\item How much value does quantified modelling add to qualitative analysis? (Coyle 2000)
		\item How can we guard against models that risk becoming plausible nonsense? (Coyle 2000)
		\item damping factor/ decreasing value of software
		\item zwei modellierungsarten: abolut und relative
	\end{itemize}
	
\end{comment}

The goal of this paper was to facilitate a better understanding of the dynamics of PaaS business models. In order to address this, a DSR approach was chosen and the resulting artifacts are a classification schema presenting the main design elements of PaaS business models as well as a qualitative system dynamics model, illustrating the complex dynamics of PaaS business models. The developed model is build upon Sterman's concept of adoption cycle and highlights two  main aspects crucial for the adoption of PaaS business models: First, each addressed customer segment needs a specific customer value proposition, based on its individual needs. Second, the interrelationships between customer segments are decisive for the platform adoption.

The results presented in the paper at hand provide a significant scientific and practical contribution. The two artefacts -- the classification schema and casual loop diagrams contribute to the knowledge and understanding of PaaS business models. In particular, the casual loop diagram explaining the dynamics of PaaS adoption, enhances existing rather structural knowledge about PaaS business models. While other authors named in section 3 "Related Work" focused on explaining the inter-relationships of business model elements, the system dynamics analysis presented in the paper at hand focuses on and explains the impact of inter-relationships of business model components with respect to their adoption. Another contribution of the research presented here is also the analysis and explanation of the inter-relationships among the various target customer segments in n-sided business models. Besides providing a general casual loop model explaining the inter-relationships among adoption processes of involved target customer segments of PaaS, the applicability of system dynamics for the analysis of n-sided business models was illustrated on a concrete example and generalized for practical use. 

The presented results provide also a contribution with practical relevance. The knowledge about typical balancing and reinforcing feedback loops presented in the generic casual loop diagrams can help PaaS providers to understand the adoption dynamics of their platform and which factors have influence on it. The evaluation example clearly demonstrates how based on the identified generic feedback loops, potential week features of existing business models can be identified.

Despite of the promising results, the study has limitations that might affect the generalizability of the results and that need to be mentioned: even though the classification schema has been evaluated in 24 case studies, the developed qualitative system dynamics model has been evaluated only in two proof of concept studies. This number is sufficient to assure rigor and relevant results, but might limit the generalizability of the CLD.

Within this paper the feedback structure -- dynamics -- of PaaS business models has been exhibited and illustrated in form of a CLD (qualitative model). Future research might include transforming this CLD into a quantitative model -- stock and flow diagram -- where the insights gained so far could be further evaluated and manifested. Moreover, such stock and flow models enable the simulation of the PaaS business model dynamics and thereby the identification of detailed high-leverage interventions and policies. Thus, simulation of business models based on the gained knowledge would be interesting further research.
