% ****************************************************************************************************
\chapter{Stock and Flow Diagram -- A quantitative Model}\label{ch:sfd}
% ****************************************************************************************************

\begin{itemize}
	\item vorerst keine CF (contact frequency) wie bei Sterman, kann im ersten versuch vernachl�ssig werden
	\item additiv oder multiplikativ
	\item annahme: total population = adopters + potentail adopters
	\item Flow von ITS zu ISV
	\item Tradeoff against the uncertainties inevitably inherent in quantification
	\item How much value does quantified modelling add to qualitative analysis? (Coyle 2000)
	\item How can we guard against models that risk becoming plausible nonsense? (Coyle 2000)
\end{itemize}

\newpage

In order to map the values of the function $f(s) \in [a_{1},a_{2}]$ to the desired interval $[b_{1},b_{2}]$, the following generic linear transformation function is used:
\begin{equation}\label{eq:ltg}
	f(t) = b_{1} + \frac{(s-a_{1})(b_{2}-b_{1})}{(a_{2}-a_{1})} \in [b_{1},b_{2}].
\end{equation}

In the particaular case of $f(s) \in [0,1]$, the formula \ref{eq:ltg} can be simplified as follows:
\begin{equation}\label{eq:lts}
	f(t) = b_{1} + s (b_{2}-b_{1}) \in [b_{1},b_{2}].
\end{equation}



%\begin{equation*}
%	s \in [a_{1},a_{2}]
%\end{equation*}

%\begin{equation*}
%	t \in [b_{1},b_{2}]
%\end{equation*}

%\begin{equation*}
%	f(s) \mapsto t
%\end{equation*}

%\begin{equation}
%	f(t) = b_{1} + \frac{(s-a_{1})(b_{2}-b_{1})}{(a_{2}-a_{1})}
%\end{equation}