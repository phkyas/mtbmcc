
\documentclass[english]{gradu3}
\usepackage{graphicx} 
\usepackage{amsmath} 
\usepackage{booktabs} 

\usepackage[bookmarksopen,bookmarksnumbered,linktocpage]{hyperref}

\addbibresource{library.bib}



\begin{document}

\title{Business Models in Cloud Computing}

\translatedtitle{Usage of the {gradu3} document class for \LaTeX\ theses}

\studyline{Kaikki suuntautumisvaihtoehdot}

\avainsanat{%
  \LaTeX,
  {gradu3},
  pro gradu -tutkielmat,
  kandidaatintutkielmat,
  k�ytt�ohje}
	
\keywords{\LaTeX, {gradu3}, Master's Theses, Bachelor's Theses, user's guide}

\tiivistelma{%
  T�m� kirjoitelma on esimerkki siit�, kuinka
  {gradu3}-tutkielmapohjaa k�ytet��n.  Se sis�lt�� my�s
  k�ytt�ohjeet ja tutkielman rakennetta koskevia ohjeita.

  Tutkielman tiivistelm� on tyypillisesti lyhyt esitys, jossa
  kerrotaan tutkielman taustoista, tavoitteesta, tutkimusmenetelmist�,
  saavutetuista tuloksista, tulosten tulkinnasta ja johtop��t�ksist�.
  Tiivistelm�n tulee olla niin lyhyt, ett� se, englanninkielinen
  abstrakti ja muut metatiedot mahtuvat kaikki samalle sivulle.
}

\abstract{%
  This document is a sample {gradu3} thesis document class
  document.  It also functions as a user manual and supplies
  guidelines for structuring a thesis document.

  The English abstract of a thesis should usually say exactly the same
  things as the Finnish tiivistelm�.
}

\author{Philipp Kyas}

%\contactinformation{Ag~C416.1, \texttt{antti-juhani.kaijanaho@jyu.fi}}

\supervisor{Pasi Tyrv�inen}
\supervisor{Andrea Giessmann}


\type{Master's Thesis} 

\maketitle
\preface

T�h�n voit kirjoittaa tutkielmasi esipuheen.  Tutkielmissa on
harvemmin esipuheita, mutta jos sen kirjoitat, pid� se lyhyen�
(enint��n sivu).

Esipuheen tulisi kertoa ennemminkin tutkielmaprosessista kuin
tutkielman sis�ll�st�.  Esimerkiksi jos tutkielman aiheen valintaan
tai tekemiseen liittyy jokin erikoinen sattumus, voit siit� kertoa
esipuheessa.  Tapana esipuheessa on my�s kiitt�� nimelt� mainiten
t�rkeimpi� tutkielman tekemisess� auttaneita ihmisi� -- ainakin
ohjaajia, puolisoa ja lapsia.  (Yleens� perhe on auttanut v�hint��n
tukemalla ja kannustamalla.)

Esipuhe kannattaa kirjoittaa min�-muodossa. Tavanomaista on my�s
allekirjoittaa se.

Jyv�skyl�ss� \today

\bigskip

Tutkielman tekij�




\begin{thetermlist}
\item[\TeX] Donald Knuthin 1977--1989 laatima eräajotyyppinen
  ladontajärjestelmä \parencite[ks.][]{knuth86:_texbook}. 
\item[\LaTeX] \TeX in \parencite[ks.][]{knuth86:_texbook} päälle
  rakennettu rakenteisten kirjoitelmien ladontaan tarkoitettu
  järjestelmä \parencite[ks.][]{lamport94:_latex}.  Siitä on nykyään
  käytössä versio \LaTeXe.
\end{thetermlist}

\mainmatter

\chapter{Johdanto}

\cite{Zott2011}

Tutkielman varsinainen teksti alkaa aina luvulla ''Johdanto''.  Sen
kirjoittamisen voi hyvin jättää aivan tutkielman kirjoitusprosessin
loppuvaiheisiin.

Johdanto kannattaa aloittaa napakasti esittämällä heti alussa
tutkielman pääväite tai tutkimuskysymys.  Tämän jälkeen kannattaa
selventää asioita määrittelemällä tarvittavat
käsitteet.\footnote{Määritelmät vasta väitteen jälkeen! Äläkä
  jaarittele johdannossa.}  Johdannossa voit myös kertoa, miksi väite
on käytännön tai tieteen (tai parhaimmillaan molempien) kannalta
relevantti ja mielenkiintoinen.  Erinomaista olisi, jos kertoisit
johdannossa lyhyesti myös, mikä on tutkielmasi kontribuutio eli mitä
sellaista tietoa tutkielmasi sisältää, jonka olet itse selvittänyt sen
sijaan että olisit sen lähteestä lukenut.  Kontribuutio voi hyvin olla
myös se, että olet itse tarkastanut jonkin lähteestä löytyneen
väitteen todenperäisyyden.  Johdannon lopuksi on tapana esitellä
lyhyesti tutkielman rakenne -- mitä missäkin luvussa käsitellään.

\begin{figure}[h]\centering
  \includegraphics[height=5cm,keepaspectratio]{opus-kissa}
  \caption[\LaTeX-oppaani \parencite{kaijanaho03:_latex_ams_latex}
  kansikuva]{\LaTeX-oppaani \parencite{kaijanaho03:_latex_ams_latex}
    kansikuva on tässä vain esimerkkinä kuvan ottamisesta mukaan
    tutkielmaan.}
  \label{fig:opus-kissa}
\end{figure}

Tämä malli käsittelee Jyväskylän yliopiston tietotekniikan laitoksella
tehtävien kandidaatintutkielmien ja pro gradu "=töiden laatimista
avustavaa \LaTeX-kirjoitelmaluokkaa gradu3 (versio \graduclsversion).
Apua sen käyttämiseen voit saada Tutkielma-TeX"-postituslistalta
(\url{http://lists.jyu.fi/mailman/listinfo/tutkielma-tex}).
Kommentteja, parannusehdotuksia ja bugiraportteja voit lähettää myös
minulle suoraan.

Tämä malli olettaa, että tunnet \LaTeX-järjestelmän käytön perusasiat.
Alkuperäinen \LaTeX-kirja \parencite{lamport94:_latex} on järjestelmän
virallinen käyttöopas.  Olen myös itse kirjoittanut aiheesta
opaskirjan \parencite{kaijanaho03:_latex_ams_latex}.\footnote{Monet
  \TeX- ja \LaTeX-kirjat käyttävät kansikuviensa aiheena kissaeläintä.
  Oman monisteeni kansikuva oli varsin abstrakti kisuli,
  ks.~kuvio~\ref{fig:opus-kissa}.}  Hyvä suomenkielinen, vapaasti
verkosta saatavilla oleva opas on \textit{Pitkänpuoleinen johdanto
  \LaTeXe:n käyttöön} \parencite{oetiker:_pitka_latex}.  Muista lukea
tämän mallin ladotun version (PDF tms) lisäksi sen \LaTeX-lähdekoodi!

Huomaa, että tämän mallin esittämät ohjeet eivät ole millään tavalla
virallisia.  Noudata aina ohjaajasi neuvoja vaikka ne poikkeaisivatkin
tämän mallin ohjeista.

\chapter{Tutkielman rakenne}

Yhteensä tutkielmassa on hyvä olla 5--9 numeroitua
lukua, siis Johdanto ja Yhteenveto mukaan lukien.  Tarvittaessa voit
käyttää alilukuja tarkempaan jäsentelyyn.

Johdannon ja Yhteenvedon väliin jääviä lukuja kutsutaan toisinaan
tutkielman \textit{käsittelyosaksi}.  Usein sen katsotaan jakaantuvan
vielä kahtia, jolloin käsittelyosa alkaa \textit{teoriaosalla} ja
päättyy joko \textit{päälauseeseen}, \textit{konstruktiiviseen osaan}
tai \textit{empiiriseen osaan}.

\section{Teoriaosa}

Tutkielman teoriaosan tarkoituksena on esitellä tutkielmassa
tarvittava teoreettinen tausta.  Tämä on syytä tehdä vähintään sillä
tarkkuudella, että tutkielman lukija pystyy pelkästään tutkielman
itsensä perusteella ymmärtämään kaikki tutkielmassa käytettävät
erityiskäsitteet ja "=menetelmät.  Monissa tutkielmissa teoria
esitellään tätä perusteellisemmin, onhan kyse opinnäytteestä eli
oppineisuuden osoittamisesta.

Teoriataustan järkevä esitys- ja käyttötapa riippuu siitä, minkä
tyyppisestä tutkimuksesta tutkielmassasi on kyse.
Matemaattis-teoreettisen työn teoriaosa on aivan eri näköinen kuin
konstruktiivisen ohjelmistonkehitystyön teoriaosa; näistä myös eroaa
olennaisesti ihmistieteellisiin traditioihin nojautuvan määrällisen
tai laadullisen tutkimustuön teoriaosa.  Muita samantyyppisiä
tutkielmia ja julkaistuja tutkimusraportteja lukemalla saat kyllä
käsityksen siitä, mitä omalta työltäsi vaaditaan.

\subsection{Lähteiden käyttö}

Teoriaosa perustuu lähes aina yksinomaan lähdekirjallisuuteen.

Muista varoa plagiointia.  Jos kopioit joko sellaisenaan tai lievästi
muutettuna (tai esimerkiksi englannista suomennettuna) tekstiä jostain
lähteestä, tee selväksi, että olet tehnyt niin.  Merkitse lainaukset
ja anna täsmällinen lähdeviite.  Jos et lainaa sanatarkasti, merkitse
tekemäsi muutokset.  Useimmissa tilanteissa on kuitenkin parempi
esittää asia omin sanoin, mieluiten useamman lähteen perusteella.
Merkitse tällöinkin käyttämäsi lähteet.

\section{Teorian jälkeen}

Teoriaosan jälkeen tulee työsi varsinainen kontribuutio:
\begin{itemize}
\item Matemaattis-teoreettisessa työssä se on yleensä jono itse
  laatimiasi määritelmiä ja lemmoja, jotka kulminoituvat työn
  päälauseen todistukseen.
\item Konstruktiivisessa työssä se on itse laatimasi tietokoneohjelma
  tai muu artefakti.
\item Empiirisessä työssä se on jotain empiiristä tutkimusmenetelmää
  soveltamalla saavutettu joukko empiirisiä tuloksia.
\end{itemize}

Tutkielmassa kontribuutio esitellään varsin tarkasti, tehdyt valinnat
perustellen.  Erityisesti matemaattis-teoreettisissa ja empiirisissä
töissä on syytä noudattaa kulloisenkin tutkimustradition käytänteitä
-- esimerkiksi ihmistieteellinen koeasetelma on kuvattava tarkasti.

\chapter{Tutkielmapohjan erityispiirteet}

Pääsääntöisesti {gradu3} käyttäytyy kuten \LaTeX in mukana
tuleva {report}-kirjoitelmaluokka.  Eroja kuitenkin on:
\begin{itemize}
\item Sinun ei tarvitse ladata {inputenc}-, {fontenc}-
  eikä {babel}-pakettia.
  \begin{itemize}
  \item Käyttämäsi merkistö sinun pitää ilmoittaa
    {\string\documentclass}-komennon optiona.  Nykyään
    {utf8} on lähes pomminvarma valinta.
  \item Jos tutkielmasi on englanninkielinen, ilmoita se
    {\string\documentclass}-komennon optiolla {english}.
  \end{itemize}
\item Jos tutkielmasi on kandidaatintutkielma, käytä
  {\string\documentclass}-komennon optiota {bachelor}.
\item Ilmoita tutkielmasi metatiedot taulukossa~\ref{tbl:metatiedot}
  esitetyillä komennoilla.  Ne tulee antaa ennen
  {\string\maketitle}-komentoa.
\begin{table}[h]\centering
  \begin{tabular}{lp{9cm}}
    \toprule
    Komento & Tarkoitus \\
    \midrule
    {\string\title}
    & Työn otsikko (älä käytä {\string\thanks}-komentoa) \\
    {\string\translatedtitle}
    & Suomenkielisen työn englanninkielinen otsikko
    englanninkielisen työn suomenkielinen otsikko\\
    {\string\studyline}
    & Suuntautumisvaihtoehtosi \\
    {\string\tiivistelma}
    & Suomenkielinen tiivistelmä \\
    {\string\abstract}
    & Englanninkielinen abstrakti \\
    {\string\avainsanat}
    & Suomenkieliset avainsanat \\
    {\string\keywords}
    & Englanninkieliset avainsanat \\
    {\string\author}
    & Kirjoittajan nimi (jos useita, anna kukin omana komentonaan -- {\string\and}-komentoa ei tueta) \\
    {\string\contactinformation}
    & Kirjoittajan yhteystiedot \\
    {\string\supervisor}
    & Tutkielman ohjaaja (jos useita, anna kukin omana komentonaan)\\
    \bottomrule
  \end{tabular}
  \caption{Metatietojen ilmoituskomennot}\label{tbl:metatiedot}
\end{table}
\item Voit \string\maketitle-komennon jälkeen halutessasi kirjoittaa
  esipuheen.  Sen otsikon saat komennolla \string\preface.
\item Mahdollisen esipuheen jälkeen voit kirjoittaa termiluettelon
  käyttämällä thetermlist-ympäristöä.  Sen sisällä voit käyttää
  \string\item[\textit{termi}]"-komentoa merkitsemään määriteltävän
  termin.
\item Käytä \string\maketitle-komennon ja mahdollisten esipuheen ja
  termiluettelon jälkeen \string\mainmatter"-komentoa.  Se laatii
  automaattisesti tarvittavat sisällys-, kuvio- ja taulukkoluettelot.
\item Komentoja \string\subsubsection, \string\paragraph{} ja
  \string\subparagraph{} ei tueta.
\item Liitteet eivät ole lukuja (\string\chapter) vaan alilukuja
  (\string\section).
\item Lähdeluettelon tekemiseen käytetään automaattisesti
  \textsc{Bib\LaTeX}-järjestelmää \parencite{biblatex-manual} ja sen
  Chicago-tyyliä \parencite{biblatex-chicago-manual}.  Tarvitset
  erillisen \textsc{Bib\TeX}"-tiedostomuodossa olevan lähdetietokannan.
  Sen laatimisessa voit käyttää apuna monia
  lähteidenhallintajärjestelmiä, mutta sen voi laatia myös käsin.
  Tarvitset myös {biber}"-ohjelman, jolla tietokanta kääntyy \LaTeX in
  ymmärtämään muotoon.\footnote{Huomaa, että perinteinen
    {bibtex}-ohjelma ei toimi {gradu3}-tutkielmapohjan käyttämän
    Chicago"-bibliografiatyylin kanssa.}
  \begin{itemize}
  \item Tämän automatiikan saa pois \string\documentclass-optiolla
    manualbib.  Tällöin joudut itse huolehtimaan lähdeluettelon
    muotoilusta.  Huomaa, että Tietotekniikan laitoksen graduissa on
    suositeltavaa käyttää Chicago-tyylistä lähdeluetteloa.
  \end{itemize}
\end{itemize}

\chapter{Yhteenveto}

Tutkielman viimeinen luku on Yhteenveto.  Sen on hyvä olla lyhyt;
siinä todetaan, mitä tutkielmassa esitetyn nojalla voidaan sanoa
johdannon väitteen totuudesta tai tutkimuskysymyksen vastauksesta.
Yhteenvedossa tuodaan myös esille tutkielman heikkoudet (erityisesti
tekijät, jotka heikentävät tutkielman tulosten luotettavuutta), ellei
niitä ole jo aiemmin tuotu esiin esimerkiksi Pohdinta-luvussa.  Tässä
luvussa voidaan myös tuoda esille, mitä tutkimusta olisi tämän
tutkielman tulosten valossa syytä tehdä seuraavaksi.

Jos Yhteenveto alkaa pitkittyä, se kannattaa jakaa kahtia niin, että
tulosten tulkinta otetaan omaksi Pohdinta-luvukseen, jolloin
Yhteenvedosta tulee varsin lyhyt ja lakoninen.

Yhteenvedon jälkeen tulee \string\printbibliography-komennolla
laadittu lähdeluettelo ja sen jälkeen mahdolliset liitteet.

\printbibliography

\appendix
\section{Siirtyminen gradu2:sta gradu3:een}

Keskeneräisen tutkielman siirtäminen gradu2:sta gradu3:een ei ole
kovin vaikeata.  Aluksi on totta kai vaihdettava
\string\documentclass-komennossa gradu2 gradu3:ksi.  Komennon
optioista suurin osa on poistettava, koska niitä ei enää tueta;
ainoastaan merkistön ilmoittava optio jää jäljelle.  Mahdollinen
kandi-optio vaihdetaan optioksi bachelor.

Taulukossa~\ref{tbl:cmdchange} on lueteltu tarvittavat
komentovaihdokset.  Viiva tarkoittaa, ettei vastaavaa komentoa ole
lainkaan.  Huomaa erityisesti uudet komennot.

\begin{table}[h]\centering
  \begin{tabular}{ll}
    \toprule
    gradu2                 & gradu3  \\
    \midrule
    ---                    & \string\maketitle \\
    ---                    & \string\supervisor \\
    \string\acmccs         & --- \\
    \string\aine           & \string\subject\\
    \string\copyrightowner & --- \\
    \string\fulltitle      & --- \\
    \string\laitos         & \string\department\\
    \string\license        & --- \\
    \string\linja          & \string\studyline\\
    \string\paikka         & --- \\
    \string\setauthor      & \string\author\\
    \string\termlist       & thetermlist-ympäristö\\
    \string\tyyppi         & \string\type\\
    \string\yhteystiedot   & \string\contactinformation\\
    \string\yliopisto      & \string\university\\
    \string\ysa            & --- \\
    \bottomrule
  \end{tabular}
  \caption{Komentomuutokset gradu2:sta gradu3:een}
  \label{tbl:cmdchange}
\end{table}

Isoin työ voi aiheutua lähdeluettelon laatimistekniikan muuttumiseen
sopeutumisesta.

\section{Harvemmin tarvittavat ominaisuudet}

Aiemmin esiteltyjen lisäksi gradu3 tarjoaa seuraavat lisäominaisuudet:
\begin{itemize}
\item \LaTeXe:n vakio-optiot draft ja final toimivat.
\item Vaikka tutkielman suomenkielisyyttä ei tarvitse erikseen
  mainita, finnish-optio toimii.
\item \string\university-komennolla voit ilmoittaa tutkielman
  kotiyliopistoksi jonkin muun kuin Jyväskylän yliopiston.
\item  \string\department-komennolla voit ilmoittaa tutkielman
  kotilaitokseksi jonkin muun kuin Tietotekniikan laitoksen.
\item \string\subject-komennolla voit ilmoittaa tutkielman
  oppiaineeksi jonkin muun kuin tietotekniikan.  Huomaa, että oppiaine
  tulisi suomenkielisissä tutkielmissa kirjoittaa genetiivimuodossa ja
  isolla alkukirjaimella (''Tietotekniikan''), englanninkielisissä
  tuktkielmissa in-preposition kanssa (''in Information Technology'').
\item \string\type-komennolla voit ilmoittaa tutkielman tyypin, jos se
  on jokin muu kuin pro gradu (oletus) tai kandidaatintutkielma
  (optiolla bachelor).
\item \string\setdate-komennolla voit asettaa päivämäärän
  haluamaksesi.  Anna komennolle kolme parametria -- päivä,
  kuukausi ja vuosi -- numeerisessa muodossa.
\item Ympäristöllä chapterquote voit laittaa luvun alkuun
  mietelauseen.  Sillä on yksi pakollinen parametri (lainauksen
  attribuutio).
\item Komento \string\graduclsdate\ sisältää käytössä olevan gradu3:n
  julkaisupäivämäärän ja \string\graduclsversion\ sen versionumeron.
\end{itemize}

\end{document}
