% ****************************************************************************************************
% Design Science Research
% ****************************************************************************************************

\begin{tikzpicture}[scale=0.75, every node/.style={scale=0.75}, node distance = 4.5cm]

\node[draw,text width=8em,text centered, rectangle,rounded corners,minimum height=4em,thick,rotate=90] (a1) {
	\begin{minipage}{8em}\centering
		Identify\\ Problem\\ \& Motivate\\
		~\\
		\textit{Emerging \ac{PaaS} market\\~\\Importance of platforms strategy and ecosystems for platform-based business models.\\~\\Complex interdependencies which are not always trivially observable.}
		\\~\\~\\~\\~\\~\\~\\
		(cf. Section \ref{ch:intro:mo})
		\\~\\~\\
	\end{minipage}
};

\node[draw,text width=8em,text centered,rectangle,rounded corners,minimum height=4em,thick,rotate=90, right of=a1] (a2) {
\begin{minipage}{8em}\centering
		Define\\ Objectives of\\ a Solution\\
		~\\
		\textit{Classify PaaS providers' business models and its main components.\\~\\ Reveal crucial interdependencies within PaaS business models elements with regards to the platform adoption, using the concept of system dynamics.}
		\\~\\~\\~\\~\\~\\
		(cf. Section \ref{ch:intro:rq})
		\\~\\~\\
	\end{minipage}
};

\node[draw,text width=8em,text centered,rectangle,rounded corners,minimum height=4em,thick,rotate=90, right of=a2] (a3) {
\begin{minipage}{8em}\centering
		Design \&\\ Development\\
		~\\~\\
		\textit{Elaboration of a classification scheme for \ac{PaaS} business models based on explorative case studies.\\~\\ Development of a system dynamics model (qualitative and quantitative) to demonstrate and simulate the adoption of platform-based business models.}
		\\~\\~\\~\\
		(cf. Chapters \ref{ch:sota}-\ref{ch:sfd})
		\\~\\~\\
	\end{minipage}
};

\node[draw,text width=8em,text centered,rectangle,rounded corners,minimum height=4em,thick,rotate=90, right of=a3,] (a4) {
\begin{minipage}{8em}\centering
		Demonstration\\
		~\\~\\~\\
		%\textit{By classifying several \ac{PaaS} business models the usability of the classification scheme is demonstrated.\\~\\ The developed system dynamics model is applied and simulated for several \ac{PaaS} business models as \acf{PoC}.}
		\textit{By classifying several \ac{PaaS} business models the usability of the classification scheme is demonstrated.\\~\\The system dynamics models -- qualitative (\ac{CLD}) as well as quantitative (\ac{SFD}) -- are illustrated, both in textual and graphical form.}
		\\~\\~\\~\\~\\
		(cf. Section \ref{ch:sota:cPaaS} as well as Chapters \ref{ch:cld} and \ref{ch:sfd})
	\end{minipage}
};

\node[draw,text width=8em,text centered,rectangle,rounded corners,minimum height=4em,thick,rotate=90, right of=a4] (a5) {
\begin{minipage}{8em}\centering
		Evaluation\\
		~\\~\\~\\
		%\textit{The developed classification scheme as well as the model is evaluated in an iterative approach trough a focus group and expert interviews.}
		\textit{The developed classification scheme as well as the models are evaluated in an iterative approach trough a focus group and expert interviews.\\~\\As a \acf{PoC} the developed quantitative system dynamics model is applied and simulated for \ac{PaaS} business models.}
		\\~\\
		(cf. Section \ref{ch:sota:cm} and Chapter \ref{ch:cld}-\ref{ch:poc})
		\\~\\
	\end{minipage}
};

\node[draw,text width=8em,text centered,rectangle,rounded corners,minimum height=4em,thick,rotate=90, right of=a5] (a6) {
\begin{minipage}{8em}\centering
		Communication\\
		~\\~\\~\\
		%\textit{As of writing this paper at hand, it is planned to write and submit a conference paper, either for the 47th \acf{HICSS} or for the 5th \acf{ICOSB}.}
		\textit{The thesis itself.\\~\\As of writing this paper at hand, it is planned to write and submit a conference paper, either for the 47th \acf{HICSS} or for the 5th \acf{ICOSB}.}
		\\~\\~\\~\\~\\~\\
	\end{minipage}
};

\node[draw,text width=5em,text centered,circle,rounded corners,minimum height=4em,thick,rotate=90, below of=a1,node distance=9.5cm] (e) {Problem-Centered Initiation};

\node[rotate=90, above left=3.51em and 1cm of a6] (ao6) {};
\node[rotate=90, above left=3.51em and 1cm of a5] (ao5) {};
\node[rotate=90, above left=3.51em and 1cm of a3] (ao3) {};
\node[rotate=90, above left=3.51em and 1cm of a2] (ao2) {};

%(a4.east)+(-1,0)

\path 
	(a1.70) edge [->,thick,>=stealth'] node [rotate=180,above left=0.2cm and 0cm of a1]{Inference} (a2.110)
	(a2.70) edge [->,thick,>=stealth'] node [rotate=180,above left=0.26cm and 0cm of a2]{Theory} (a3.110)
	(a3.70) edge [->,thick,>=stealth'] node [rotate=180,above left=0.26cm and 0cm of a3]{How to Knowledge} (a4.110)
	(a4.70) edge [->,thick,>=stealth'] node [rotate=180,above left=0.26cm and 0cm of a4]{Metrics, Analysis Knowledge} (a5.110)
	(a5.70) edge [->,thick,>=stealth'] node [rotate=180,above left=0.26cm and 0cm of a5]{Disciplinary Knowledge} (a6.110)
	(e.north) edge [->,thick,>=stealth'] (a1.south)
	
	(a6.north) edge [-,thick,>=stealth'] (ao6.south)
	(a5.north) edge [-,thick,>=stealth'] (ao5.south)
	(a3.north) edge [-,thick,>=stealth'] (ao3.south)
	(ao6.south) edge [-,thick,>=stealth'] node [rotate=90, above left=3.18em and 0.1cm of a4]{Process Iteration} (ao2.south)
	(ao3.south) edge [->,thick,>=stealth'] (a3.north)
	(ao2.south) edge [->,thick,>=stealth'] (a2.north);

\end{tikzpicture}

