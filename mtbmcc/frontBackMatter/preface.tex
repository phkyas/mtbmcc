\preface

T�h�n voit kirjoittaa tutkielmasi esipuheen.  Tutkielmissa on
harvemmin esipuheita, mutta jos sen kirjoitat, pid� se lyhyen�
(enint��n sivu).

Esipuheen tulisi kertoa ennemminkin tutkielmaprosessista kuin
tutkielman sis�ll�st�.  Esimerkiksi jos tutkielman aiheen valintaan
tai tekemiseen liittyy jokin erikoinen sattumus, voit siit� kertoa
esipuheessa.  Tapana esipuheessa on my�s kiitt�� nimelt� mainiten
t�rkeimpi� tutkielman tekemisess� auttaneita ihmisi� -- ainakin
ohjaajia, puolisoa ja lapsia.  (Yleens� perhe on auttanut v�hint��n
tukemalla ja kannustamalla.)

Esipuhe kannattaa kirjoittaa min�-muodossa. Tavanomaista on my�s
allekirjoittaa se.

Jyv�skyl�ss� \today

\bigskip

Tutkielman tekij�