% ****************************************************************************************************
% abstract
% ****************************************************************************************************
\pdfbookmark[1]{Abstract}{Abstract}
\chapter*{Abstract}
\thispagestyle{empty}
%\addcontentsline{toc}{chapter}{Abstract}


\myName\\
\myTitle ~--- \mySubtitle\\
\myLocation : \myUni , \myTime , \pageref{LastPage} p.\\
\myDegree , \mySubject\\
Supervisors: \mySupervisor\\

\noindent
Cloud computing and its inherent components \ac{IaaS}, \ac{PaaS}, as well as \ac{SaaS} introduce a new software paradigm within the \ac{IT} domain. The way how cloud-based software respectively services are developed, deployed, managed, distributed, consumed, and priced is fundamentally redefined. Eminently the intermediary \ac{PaaS} layer provides new possibilities particularly with regards to how software is developed and deployed. However, \ac{PaaS} offers are characterized by two- or even multi-sided business models with diverse business model elements and their complex interdependencies are not always trivially observable. Thus the adoption process of \ac{PaaS} business models is elusive and designing adequate business models remains challenging. The thesis at hand contributes to this problem threefold: First, a classification scheme for \ac{PaaS} business models is developed in order to reveal the general \ac{PaaS} business model elements. Second, the relationships between these elements are investigated and illustrated by means of a \ac{CLD}. And third, the \ac{CLD} is transformed (quantified) into a \ac{SFD} so as to simulate the adoption of \ac{PaaS} business models.\\

\noindent
Keywords: \myKeywords