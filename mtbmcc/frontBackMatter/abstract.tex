% ****************************************************************************************************
% Abstract
% ****************************************************************************************************
\pdfbookmark[1]{Abstract}{Abstract}
\chapter*{Abstract}
\thispagestyle{empty}
%\addcontentsline{toc}{chapter}{Abstract}


\begin{comment}
	Author: \myName\\
	Title: \myTitle ~--- \mySubtitle\\
	\myLocation : \myUni , \myTime , \pageref{LastPage} p.\\
	\myDegree , \mySubject\\
	Supervisors: \mySupervisor\\
\end{comment}

%Cloud computing and its inherent components \ac{IaaS}, \ac{PaaS}, as well as \ac{SaaS} introduce a new software paradigm within the \ac{IT} domain. The way how cloud-based solutions are developed, deployed, managed, distributed, used, and priced is fundamentally redefined. Eminently the intermediary \ac{PaaS} layer provides new possibilities particularly with regards to how software is developed and deployed. However, \ac{PaaS} offers are characterized by two- or even multi-sided business models with diverse business model elements and their complex interdependencies are not always trivially observable. Thus the adoption process of \ac{PaaS} business models is elusive and designing adequate business models remains challenging. The thesis at hand contributes to this problem threefold: First, a classification scheme for \ac{PaaS} business models is developed in order to reveal the general \ac{PaaS} business model elements. Second, the relationships between these elements are investigated and illustrated by means of a \ac{CLD}. And third, the \ac{CLD} is transformed (quantified) into a \ac{SFD} so as to simulate the adoption of \ac{PaaS} business models. 

%Cloud computing and its inherent components \ac{IaaS}, \ac{PaaS}, and \ac{SaaS} introduce a new software paradigm within the \ac{IT} domain. The way cloud-based solutions are developed, deployed, managed, distributed, used, and priced is fundamentally redefined. Eminently the intermediary \ac{PaaS} layer provides new possibilities particularly with regards to how software is developed and deployed. However, \ac{PaaS} offers are characterized by two-sided or even multi-sided business models and their complex interdependencies are not always trivially observable. Thus, the adoption process of \ac{PaaS} business models is elusive and designing adequate business models remains challenging.

%However, there has been little research into understanding the dynamics of \ac{PaaS} business models and how these business models should be designed to establish a flourishing ecosystem around these platforms. Using design science research (DSR) the main elements of \ac{PaaS} business models as well as their relationships have been identified by investigating 23 PaaS business models. The resulting classification scheme as well as systems dynamics models -- in form of a \ac{CLD} and \ac{SFD} -- facilitate a better understanding of the adoption dynamics of \ac{PaaS} business models. By focusing on the adoption dynamics of \ac{PaaS} business models, this research goes beyond previous approaches for studying \ac{PaaS} business models.

\ac{PaaS} solutions are fundamentally changing the way cloud-based software is developed, deployed, managed, distributed, used, and priced. However, \ac{PaaS} platforms are often characterized by two-sided, and in general by multi-sided, business models and their complex interdependencies are not always immediately observable. Thus, the adoption process of \ac{PaaS} business models is elusive and designing adequate business models remains challenging. To date, there has been little research into understanding the dynamics of \ac{PaaS} business models and how these business models should be designed to establish a flourishing ecosystem around these platforms. Using \ac{DSR} in combination with case study research as well as system dynamics theory, the main design elements of \ac{PaaS} business models as well as their qualitative and quantitative interdependencies have been identified by investigating 23 PaaS business models. The resulting \ac{DSR} artifacts consist in a classification scheme as well as system dynamics models in the form of a \ac{CLD} and \ac{SFD}. These facilitate a better understanding of the adoption dynamics of \ac{PaaS} business models. Moreover, the quantitative \ac{SFD} serves to simulate the adoption of \ac{PaaS} business models by means of a simulation model. By focusing on the adoption dynamics of \ac{PaaS} business models, this research goes beyond previous approaches for studying \ac{PaaS} business models.


\vspace*{5mm}

\footnotesize
\noindent \hspace{-3mm}
\begin{tabular}{L{.23\textwidth}L{.75\textwidth}}
	Author: & \myName \\
	Title: & \myTitle ~--- \mySubtitle \\
	Keywords: & \myKeywords \\
	Supervisors: &\mySupervisor \\
	University: & \myUni \\
	Faculty: & \myFaculty \\
	Department: & \myDepartment \\
	Field of Study: & \myDegree \\ 
	Subject: & \mySubject\\
	Location: & \myLocation \\
	Year: & \myTime \\
	Number of Pages: & \pageref{LastPage}
\end{tabular}

\normalsize