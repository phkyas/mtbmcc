% ****************************************************************************************************
% tables
% ****************************************************************************************************

\renewcommand{\listfigurename}{Figures}
\listoffigures
\thispagestyle{empty}
\addcontentsline{toc}{chapter}{Figures}

\renewcommand{\listtablename}{Tables}
\listoftables
\thispagestyle{empty}
\addcontentsline{toc}{chapter}{Tables}

\renewcommand{\contentsname}{Tables of Contents}
\tableofcontents
\thispagestyle{empty}
\addcontentsline{toc}{chapter}{Tables of Contents}

\chapter*{Index of Concepts}
\thispagestyle{empty}
\addcontentsline{toc}{chapter}{Index of Concepts}

\begin{acronym}[PaaS]
		\acro{DRY}{Don't Repeat Yourself}
		\acro{API}{Application Programming Interface}
		\acro{UML}{Unified Modeling Language}
\end{acronym}



% \ac{KDE}  Gibt bei der ersten Verwendung die Langform mit der Abkürzung in Klammern aus, ab dann stets die Kurzform.
% \acs{KDE} Gibt die Abkürzung aus.
% \acf{KDE} Gibt die Langform und die Kurzform aus.
% \acl{KDE} Gibt nur die Langform ohne die Kurzform aus.




