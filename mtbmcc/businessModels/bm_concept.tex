% ****************************************************************************************************
% Business Model Concept
% ****************************************************************************************************
\begin{longtable}{|L{\column}|L{\column}|L{\column}|L{\column}|}
	
	\hline
	\endfirsthead
	\hline  
	\multicolumn{4}{|l|}{\textit{continued from previous page (\nameref{bm:concept})}}\\ 
	\hline
	\endhead
	\hline
	\multicolumn{4}{|r|}{\textit{continued on next page}}\\
	\hline
	\endfoot
	\hline
	\caption[Business Model Conceptualization]{Business Model Conceptualization adapted from \citet[p. 54]{Johnson2008}}
	\label{bm:concept}
	\endlastfoot
	
	\multicolumn{4}{|>{\columncolor[gray]{0.95}}c|}{\large \textbf{Customer Value Proposition (CVP)}}\\ \hline
	
	\textbf{Target customer} &
	\textbf{Job to be done} to solve an important problem or fulfill an important need for the target customer &
	\multicolumn{2}{|L{\columnT}|}{\textbf{Offering}, which satisfies the problem or fulfills the need. This is defined not only by what is sold but also by how it's sold.} \\ \hline

% KEY RESOURCES
\multicolumn{2}{|L{\columnT}|}{\large \textbf{Key Resources} \normalsize needed to deliver the customer value proposition profitably. Might include:
\begin{itemize}[leftmargin=*, parsep=0pt, topsep=0pt, itemsep=0pt]
	\item \textbf{People}
	\item \textbf{Technology, products}
	\item \textbf{Equipment}
	\item \textbf{Information}
	\item \textbf{Channels}
	\item \textbf{Partnerships, alliances}
	\item \textbf{Brand}\vspace{-\baselineskip} 
\end{itemize}
} &
\multicolumn{2}{|L{\columnT}|}{\large \textbf{Key Processes} \normalsize as well as rules, metrics, and norms, that make the profitable delivery of the customer value proposition repeatable and scalable. Might include:
\begin{itemize}[leftmargin=*, parsep=0pt, topsep=0pt, itemsep=0pt]
	\item \textbf{Processes:} design, product development, sourcing, manufacturing, marketing, hiring and training, IT
	\item \textbf{Rules and metrics:} margin requirements for investment, credit terms, lead times, supplier terms
	\item \textbf{Norms:} opportunity size needed for investment, approach to customers and channels\vspace{-\baselineskip} 
\end{itemize}} \\ \hline

% PROFIT FORMULA
\multicolumn{4}{|L{\columnF}|}{\large \textbf{Profit Formula} \normalsize
\begin{itemize}[leftmargin=*, parsep=0pt, topsep=0pt, itemsep=0pt]
		\item \textbf{Revenue model:} How much money can be made: price x volume. Volume can be thought of in terms of market size, purchase frequency, ancillary sales, etc.
		\item \textbf{Cost structure:} How costs are allocated: includes cost of key assets, direct costs, indirect costs, economies of scale.
		\item \textbf{Margin model:} How much each transaction should net to achieve desired profit levels. 
		\item \textbf{Resource velocity:} How quickly resources need to be used to support target volume. Includes lead times, throughput, inventory turns, asset utilization, and so on.\vspace{-\baselineskip} 
\end{itemize} 
}\\ 
\end{longtable}



