% ****************************************************************************************************
% config
% ****************************************************************************************************
\usepackage[left=3cm,right=3cm,top=3cm,bottom=2.5cm]{geometry}
\usepackage[latin1]{inputenc}
\usepackage{babel}
\usepackage[thref,hyperref]{ntheorem}
\usepackage{xspace} % to get the spacing after macros right
\usepackage[sc]{mathpazo} % font Palatino
\linespread{1.05}  % Palatino needs more leading (space between lines)
\usepackage[T1]{fontenc}
\usepackage{textcase}
\usepackage{graphicx}
\usepackage{lastpage}
\usepackage{url}
\usepackage{apacite}
\usepackage[longnamesfirst]{natbib}
\usepackage{subcaption}
\newcommand{\mycite}[4]{(\citealp[#2]{#1} and \citealp[#4]{#3})}
\usepackage{fixltx2e} % fixes some LaTeX stuff
\usepackage{acronym}%[printonlyused]
\usepackage[nottoc,notlot,notlof]{tocbibind}
\usepackage{amsmath}
\usepackage{epstopdf}
\usepackage{verbatim}  % u.a. multiline comments
% ****************************************************************************************************
% TABLES
% ****************************************************************************************************
\usepackage{booktabs}
\usepackage{longtable}
\usepackage{rotating}
\usepackage{tabularx}
\usepackage{multirow}
\usepackage{calc}
\usepackage{enumitem}
\usepackage{colortbl}
\usepackage{nameref}
\usepackage{diagbox}
\newcolumntype{L}[1]{>{\raggedright\arraybackslash}p{#1}} % linksbündig mit Breitenangabe
\newcolumntype{C}[1]{>{\centering\arraybackslash}b{#1}} % zentriert mit Breitenangabe
\newcolumntype{R}[1]{>{\raggedleft\arraybackslash}p{#1}} % rechtsbündig mit Breitenangabe
\definecolor{lightgray}{gray}{.95}
\newlength{\column}
\setlength{\column}{0.25\textwidth - 2\tabcolsep}
\newlength{\columnT}
\setlength{\columnT}{2\column + 2\tabcolsep}
\newlength{\columnF}
\setlength{\columnF}{4\column + 6\tabcolsep}
% ****************************************************************************************************
% headings
% ****************************************************************************************************
\usepackage{titlesec}
% \titleformat{command}{heading}{number}{distance}{code before}
\titleformat{\chapter}{\Large}{\textbf{\Large\thechapter}}{15pt}{\textbf}
\titleformat{\section}{\large}{\textbf{\large\thesection}}{10pt}{\textbf}
\titleformat{\subsection}{\normalsize}{\textbf{\normalsize\thesubsection}}{5pt}{\textbf}
% ****************************************************************************************************
% Theorem
% ****************************************************************************************************
\newtheorem*{MRQ}{Main Research Question}
\newtheorem{SRQ}{Sub-Research Question}
% ****************************************************************************************************
% page numbering
% ****************************************************************************************************
\usepackage{fancyhdr}
\pagestyle{fancy}
\fancyhf{}
\fancyhead[C]{\thepage}
\renewcommand{\headrulewidth}{0pt} %obere Trennlinie
\makeatletter
\let\ps@plain\ps@fancy
\makeatother
% ****************************************************************************************************
% hyperref
% ****************************************************************************************************
\usepackage[pdftex]{hyperref}
\hypersetup{
	breaklinks=true,
	pdfpagemode=UseNone, 
	pageanchor=true, 
	pdfpagemode=UseOutlines,
	plainpages=false, 
	bookmarksnumbered, 
	bookmarksopen=false, 
	%bookmarksopenlevel=1,
  hypertexnames=true, 
	pdfhighlight=/O,
  urlcolor=webbrown, 
	linkcolor=RoyalBlue, 
	citecolor=webgreen, 
	pdftitle={\myTitle},
	pdfauthor={\textcopyright\ \myName, \myUni, \myFaculty},
	pdfsubject={},
  pdfkeywords={},
  pdfcreator={pdfLaTeX},
  pdfproducer={LaTeX with hyperref and scrreprt},
	hidelinks
}
% ****************************************************************************************************
% TikZ
% ****************************************************************************************************
\usepackage{tikz}
\usetikzlibrary{calc,arrows}
\usetikzlibrary{fit}
\usetikzlibrary{positioning}
% ****************************************************************************************************
% Listings
% ****************************************************************************************************
\usepackage{listings}
\lstset{language=XML,
    basicstyle=\footnotesize,
    commentstyle=\ttfamily,
    showstringspaces=false,
    frameround=ftff,
    frame=single,
		captionpos=b, 
    belowcaptionskip=.75\baselineskip
} 